%%%%%%%%%%%%%%%%%%%%%%%%%%%%%%%%%%%%%%%%%
%
% PSI Chair Thesis Template
% Version 20200913
%
% based on MastersDoctoralThesis.cls
% Version 2.5 (27/8/17)
%
% which was obtained from:
% http://www.LaTeXTemplates.com
%
% Version 2.x major modifications by:
% Vel (vel@latextemplates.com)
%
% This template is based on a template by:
% Steve Gunn (http://users.ecs.soton.ac.uk/srg/softwaretools/document/templates/)
% Sunil Patel (http://www.sunilpatel.co.uk/thesis-template/)
%
% License of this guide and the template
% CC BY-SA 4.0 (http://creativecommons.org/licenses/by-sa/4.0/)
%
% Exception 1: Some excerpts, figures, and tables that have been taken
% from the literature (denoted with a citation in the caption) are not
% covered by the above license. Permission to re-use and distribute
% these excerpts, figures, and tables must be obtained from the
% respective holder of the copyrights.
%
% Exception 2: Chapter 1 and Appendix C are based on content from the
% MastersDoctoralThesis template mentioned above, which is licensed under
% CC BY-SA 3.0 (http://creativecommons.org/licenses/by-nc-sa/3.0/)
%
% License of the PSIThesis.cls class file:
% LPPL v1.3c (http://www.latex-project.org/lppl)
%
%%%%%%%%%%%%%%%%%%%%%%%%%%%%%%%%%%%%%%%%%

%----------------------------------------------------------------------------------------
%	PACKAGES AND OTHER DOCUMENT CONFIGURATIONS
%----------------------------------------------------------------------------------------
\PassOptionsToPackage{english,ngerman}{babel}
\documentclass[
11pt, % The default document font size is 11 (recommended), options: 10pt, 11pt, 12pt
%oneside, % Two-side layout is recommended; uncomment to switch to one-sided
english, % replace with ngerman for German; not fully supported so far -- requires changes elsewhere
singlespacing, % Single line spacing (recommended), alternatives: onehalfspacing or doublespacing
%draft, % Uncomment to enable draft mode (no pictures, no links, overfull hboxes indicated)
%nolistspacing, % If the document is onehalfspacing or doublespacing, uncomment this to set spacing in lists to single
%liststotoc, % Uncomment to add list of figures/tables/etc to table of contents (not recommended)
%toctotoc, % Uncomment to add the main table of contents to the table of contents (not recommended)
parskip, % add space between paragraphs (recommended)
nohyperref, % do not load the hyperref package (is loaded in setup.tex)
%headsepline, % print a horizontal line under the page header
consistentlayout, % layout of declaration, abstract and acknowledgements pages matches the default layout
%final, % Uncomment to hide all todo notes
]{PSIThesis} % The class file specifying the document structure

% version of the guide
\def\tversion{v20200913}

% long-term stable URL to the thesis guide
\def\doiurl{https://doi.org/10.20378/irb-48428}
\def\githuburl{https://github.com/UBA-PSI/psi-thesis-guide}

\input{misc/setup.tex} % Load the settings from Misc/setup.tex
\input{misc/commands.tex} % Load the custom commands from Misc/commands.tex

% Uncomment this command to make all links black:
%   useful for printing on black-white printers that do a
%   poor job at rasterizing colored text properly
%\hypersetup{colorlinks=false}
\addbibresource{clean-lit.bib} % The filename of the bibliography

\usepackage{bytefield}
\usepackage{calc}
%\usepackage{minted}

%----------------------------------------------------------------------------------------
%	THESIS INFORMATION
%----------------------------------------------------------------------------------------
\newcommand{\thesistype}{Bachelor} % type of your thesis (Bachelors, Masters, Doctoral ...)

%%% CHANGE THIS:
% Your thesis title, this is used in the title and abstract, print it elsewhere with \ttitle
\thesistitle{Alternative Approaches for Virtual Memory Management}

% date to be printed on the title, this will automatically update and be in the correct format
% If any changes to this format (Month JJJJ) are necessary the definition can be found in line 337
% of misc/setup.tex
\def\tdate{\monthyeardate\today}

%%% CHANGE THIS:
% Your name, this is used in the title page, print it elsewhere with \authorname
\author{Max Meidinger}

% Your supervisor's name, this is used in the title page, print it elsewhere with \supname
\supervisor{Prof. Dr. Michael Engel}

% Your university's name and URL, this is used in the title page, print it elsewhere with \univname
\university{\href{https://www.uni-bamberg.de/en/}{University of Bamberg}}

% Your research group's name and URL, this is used in the title page, print it elsewhere with \groupname
\group{\href{https://www.uni-bamberg.de/en/psi/}{Privacy and Security in Information Systems Group}}

% Your department's name and URL, this is used in the title page, print it elsewhere with \deptname
\department{department not used}

% Your faculty's name and URL, this is used in the title page, print it elsewhere with \facname
% TODO: insert *your* degree program in the \faculty command below
% Applied Computer Science
% Computing in the Humanities
% Information Systems
% International Information Systems Management
% International Software Systems Science
% Software Systems Science
% Education in Business and Information Systems
\faculty{Software Systems Science Degree Program in the\\ \href{https://www.uni-bamberg.de/en/wiai/}{Faculty of Information Systems and Applied Computer Sciences}}

% Your address, this is not currently used anywhere in the template, print it elsewhere with \addressname
\addresses{address not used}

% Your subject area, this is not currently used anywhere in the template, print it elsewhere with \subjectname
\subject{subject not used}

% Keywords for your thesis, this is not currently used anywhere in the template, print it elsewhere with \keywordnames
\keywords{keywords not used}



% Stolen from https://tex.stackexchange.com/questions/105995/is-there-a-ready-solution-to-typeset-a-diff-file

\usepackage[svgnames]{xcolor}
  \definecolor{diffstart}{named}{Grey}
  \definecolor{diffincl}{named}{Green}
  \definecolor{diffrem}{named}{OrangeRed}

\usepackage{listings}
  \lstdefinelanguage{diff}{
    basicstyle=\ttfamily\small,
    morecomment=[f][\color{diffstart}]{@@},
    morecomment=[f][\color{diffincl}]{+\ },
    morecomment=[f][\color{diffrem}]{-\ },
  }


  % more piracy https://gist.github.com/AntonLydike/e339c3c3a4dcab8bc3c620b3fa436cda


% RISC-V Assembler syntax and style for latex lstlisting package
%
% These are risc-v commands as per our university (University Augsburg, Germany) guidelines.
%
% Author: Anton Lydike
%
% This code is in the public domain and free of licensing

% language definition
\lstdefinelanguage[RISC-V]{Assembler}
{
  alsoletter={.}, % allow dots in keywords
  alsodigit={0x}, % hex numbers are numbers too!
  morekeywords=[1]{ % instructions
    lb, lh, lw, lbu, lhu,
    sb, sh, sw,
    sll, slli, srl, srli, sra, srai,
    add, addi, sub, lui, auipc,
    xor, xori, or, ori, and, andi,
    slt, slti, sltu, sltiu,
    beq, bne, blt, bge, bltu, bgeu,
    j, jr, jal, jalr, ret,
    scall, break, nop
  },
  morekeywords=[2]{ % sections of our code and other directives
    .align, .ascii, .asciiz, .byte, .data, .double, .extern,
    .float, .globl, .half, .kdata, .ktext, .set, .space, .text, .word
  },
  morekeywords=[3]{ % registers
    zero, ra, sp, gp, tp, s0, fp,
    t0, t1, t2, t3, t4, t5, t6,
    s1, s2, s3, s4, s5, s6, s7, s8, s9, s10, s11,
    a0, a1, a2, a3, a4, a5, a6, a7,
    ft0, ft1, ft2, ft3, ft4, ft5, ft6, ft7,
    fs0, fs1, fs2, fs3, fs4, fs5, fs6, fs7, fs8, fs9, fs10, fs11,
    fa0, fa1, fa2, fa3, fa4, fa5, fa6, fa7
  },
  morecomment=[l]{;},   % mark ; as line comment start
  morecomment=[l]{\#},  % as well as # (even though it is unconventional)
  morestring=[b]",      % mark " as string start/end
  morestring=[b]'       % also mark ' as string start/end
}

% usage example:

% define some basic colors
\definecolor{mauve}{rgb}{0.58,0,0.82}

\lstset{
  % listings sonderzeichen (for german weirdness)
  literate={ö}{{\"o}}1
           {ä}{{\"a}}1
           {ü}{{\"u}}1,
  basicstyle=\tiny\ttfamily,                    % very small code
  breaklines=true,                              % break long lines
  commentstyle=\itshape\color{green!50!black},  % comments are green
  keywordstyle=[1]\color{blue!80!black},        % instructions are blue
  keywordstyle=[2]\color{orange!80!black},      % sections/other directives are orange
  keywordstyle=[3]\color{red!50!black},         % registers are red
  stringstyle=\color{mauve},                    % strings are from the telekom
  identifierstyle=\color{teal},                 % user declared addresses are teal
  frame=l,                                      % black line on the left side of code
  language=[RISC-V]Assembler,                   % all code is RISC-V
  tabsize=4,                                    % indent tabs with 4 spaces
  showstringspaces=false                        % do not replace spaces with weird underlines
}

% EOP - End of piracy
%----------------------------------------------------------------------------------------
%	END OF THESIS INFORMATION
%----------------------------------------------------------------------------------------

\setcounter{tocdepth}{3}

\begin{document}

\selectlanguage{english}

\frenchspacing % do not add additional hspace after end of sentence full stop dot.

\frontmatter % Uses roman page numbering style (i, ii, iii, iv...) for the pre-content pages

\hypersetup{urlcolor=black}

\include{misc/titlepage} % Typeset the titlepage

\hypersetup{urlcolor=ubblue80}


%----------------------------------------------------------------------------------------
%	QUOTATION
%----------------------------------------------------------------------------------------

% \vspace*{0.2\textheight}

% \noindent\enquote{\itshape Thanks to my solid academic training, today I can write hundreds of words on virtually any topic without possessing a shred of information, which is how I got a good job in journalism.}\bigbreak

% \hfill Dave Barry


%----------------------------------------------------------------------------------------
%	ABSTRACT PAGE
%----------------------------------------------------------------------------------------

\begin{abstract}
  %\addchaptertocentry{\abstractname}
  % uncomment to add the abstract to the table of contents (not recommended)
  Hier sind alle todos die für das ganze paper gelten:
  \begin{itemize}
    \item formatting single page or double page
    \item call it faulting, failing, missing address?
  \end{itemize}
\end{abstract}


%----------------------------------------------------------------------------------------
%	ACKNOWLEDGEMENTS
%----------------------------------------------------------------------------------------

%\begin{acknowledgements}
% %\addchaptertocentry{\acknowledgementname}
% Add the acknowledgements to the table of contents (not recommended)
%
%The acknowledgments and the people to thank go here.
%\end{acknowledgements}


%----------------------------------------------------------------------------------------
%	TABLE OF CONTENTS
%----------------------------------------------------------------------------------------

\cleardoublepage

% Table of Contents uses a wider layout than the main content
\newgeometry{
  head=13.6pt,
  top=27.4mm,
  bottom=27.4mm,
  inner=24.8mm,
  outer=24.8mm,
  marginparsep=0mm,
  marginparwidth=0mm,
}
{
  \hypersetup{linkcolor=black}
  \tableofcontents % Prints the ToC entries
}
\restoregeometry

%----------------------------------------------------------------------------------------
%	DEDICATION
%----------------------------------------------------------------------------------------

% \dedicatory{For/Dedicated to/To my\ldots}


%----------------------------------------------------------------------------------------
%	THESIS CONTENT - CHAPTERS
%----------------------------------------------------------------------------------------
\mainmatter % From here on, numeric (1,2,3...) page numbering
\pagestyle{thesis} % Return the page headers back to the "thesis" style

% Define some commands to keep the formatting separated from the content
\newcommand{\keyword}[1]{\textbf{#1}}
\newcommand{\tabhead}[1]{\textbf{#1}}
\newcommand{\code}[1]{\texttt{#1}}
\newcommand{\file}[1]{\texttt{#1}}
\newcommand{\option}[1]{\texttt{\itshape#1}}

% Figures will automatically be searched for in the Figures subdirectory
\graphicspath{{./figures/}{./examples/}}

%%% CHANGES NEEDED HERE
%
% Include the chapters of the thesis as separate files from the Chapters folder
% Uncomment the lines as you write the chapters
% Mind the \input instead of the \include here, that change is necessary for the appendix formatting
% Due to the \input command you also need to provide the .tex file ending



\chapter{Introduction} % Main chapter title
% TODO CHECK 2-3 Seiten
% TODO CHECK Keine Ergebnisse
% TODO CHECK Keine Definitionen
% TODO CHECK Alles relevante vorhanden



% -------------------------------------------------------------------------------------------------
%                                             Structure
% -------------------------------------------------------------------------------------------------



% Motivation for VM

% Description of Problems with VM

% Description of Approach

% Proof of concept
% Description of Contents









% -------------------------------------------------------------------------------------------------
% -------------------------------------------------------------------------------------------------
% -------------------------------------------------------------------------------------------------
% -------------------------------------------------------------------------------------------------
% Motivation
% Most Computer use Virtual Memory

% THIS SECTION may primarily be about some facts about VM

%Quick Facts
% Intel 5 Level Paging intelTLBsPagingStructureCaches2008
% Sv57 RISC-V Paging RISCVInstructionSet


% Plethora of interfaces suggesting performance problems

% [ A look at several ]
% implemented on either the hardware or software side of the interface. The myriad of design choices and incompatible hardware mechanisms suggests potential performance problems, especially since increasing numbers of systems (even embedded systems) are using memory management. A comparative study of the implementation choices in virtual memory should therefore aid system-level designers.
\label{Intro}

\todo{CITE Architectural and operating system support for virtual memory}
Virtual memory is used by computer systems at all scales. Everything from systems running in huge data
centers to embedded computers rely on virtual memory.

% Relevance: Most computers today use a page-based virtual memory approach
Current radix page tables can get up to 5 levels deep \cite{intel5LevelPaging5Level2017}. Every level
adds another memory access to the page table lookup.\\
Caching structures can alleviate the cost by storing complete or partial translation results \cite{van2002memory}.
But especially large-memory applications, virtualized environments and graph-processing applications
suffer extensively from address translation costs \cite{zagieboylo2020cost}.


% -------------------------------------------------------------------------------------------------
% Problems with VM

% SOME NUMBERS PLEASE -> [ A look at several ...] abstract
% Previous approaches aimed to reduce the memory footprint or optimize cache accesses, or reduce page table
% walks to a minimum, this approach aims to eliminate paging structures completely

% Very short summarization of the way page-based virtual memory works, ending with the amount of memory references
% barrTranslationCachingSkip -> memory references keep on increasing as memory requirements increase

While previous work tried to either reduce the number of memory accesses \todo{CITE Lidtke ,Skip no walk, single hit uppsala paper},
decrease the occurrence of a TLB miss and decrease the latency of handling a TLB Miss, this paper
presents an approach to create a virtual memory scheme not requiring paging structures at all.\\


% TODO Problem Statement: Are paging structures even necesarry or can virtual memory purely be realized
%       by a simple software-defined function that refills the tlb?
%

% Conclusion on previous Work -> Still need a memory access

% The Idea -> Getting rid of any memory access in favor of simple arithmetic operation (eg. hash functions)
% statefull -> stateless

% -------------------------------------------------------------------------------------------------
% -------------------------------------------------------------------------------------------------
% Description of Approach
% TODO Very short: How the TLB is used

% The approach is not directly the TLB filling, thats just one
% implementation of the process.

% The main idea is to replace page table walks with simple function calls

Current hardware-supported virtual memory systems suffer from inflexibility and costly memory accesses \cite{jacob1998look}.\\
Software-based approaches are a lot more flexible, but are, so far, based on page table lookups as well.

% Soruce: Virtual memory: issues of implementation

\begin{itemize}
    \item Software Managed TLB
    \item Not based on page tables but a simple function to reduce memory accesses to a minimum
\end{itemize}


The RISC-V Instruction Set manual warns that ''[...]Software TLB refills are a performance bottleneck[...]''\cite{RISCVInstructionSet},
but also suggests that they may be realized by implementing a machine-mode trap handler as an extension to the
most-priviledged machine mode. This paper will explore just that.
% Auch -> Proof of concept
% Und: Keine Page table walks
% Discussion: -> A look at several ...


% WHAT DOES THIS PAPER DO
% -> SW TLB Handling Proof Of concept on riscv
% -> Use that TLB handling to realize a Segmented vm design not using any page tables


% -------------------------------------------------------------------------------------------------
% -------------------------------------------------------------------------------------------------
% Proof of concept


% -------------------------------------------------------------------------------------------------
% -------------------------------------------------------------------------------------------------
% Description of Contents


% Inhaltsbeschreibung
Kapitel 2 gibt eine Übersicht über ein bisschen Historie und die Grundlagen von Virtuellen
Speichersystemen. Das dort zusammengefasste Wissen stellt die Grundlage für den restlichen Teil der
Arbeit dar.\\
Kapitel 3 beschäftigt sich mit verwandten Arbeiten die sich auch mit der Optimierung des VM Systems
beschäftigen.\\
Kapitel 4 beschreibt die Theoretische Ausarbeitung der \textit{softtlb} Idee und stellt so mit die
Grundlage für die Implementierung da.\\
Die Implementierung wird in Kapitel 5 beschrieben. Dort wird näher auf die Eigenheiten der
Programmierplattformen eingegangen und der Implementationsprozess schrittweiße dargestellt.
Dort findet sich auch eine Übersicht der Debuggingtechniken die zur Verifikation und Fehlersuche
verwendet wurden.\\
Das sechste Kapitel setzt sich kritisch mit dem erziehlten Stand der theoretischen Ausarbeitung und
der Implementation außeinander und beleuchtet das Ergebnis im lichte typischer Requirements an
andere Virtuelle Speichersysteme. Dort schließt sich noch eine Diskussion über eine weitere
Vertiefung des Ansatzes an.\\
Im siebten Kapitel werden noch Vorschläge für auf diese Arbeit aufbauende Arbeiten gemacht. Diese
nehmen noch Bezug auf zuvor ausgeschlossene Themen wie Hardwareoptimierungen der Idee.\todo{future  work:
    Hardware stateless hash mmu, der meine Idee in hardware realisiert}
Das achte und letzte Kapitel gibt eine zusammenfassende Einschätzung der Arbeit ab.\\
% -------------------------------------------------------------------------------------------------






\cite{zagieboylo2020cost} has nice numbers in introduction for cost of VM
\chapter{Fundamentals} % Main chapter title
\label{chap:fund}

% Hier muss alles rein was benötigt wird um die Arbeit zu verstehen.
%   Man kann ja sicherlich grundlegendes Verständis von Recherstrukturen und Organisation vorraussetzten
%   Was also sollte definitiv nochmal erklärt werden?
%
%   - Virtual Memory -> Und vor allem die Kosten, das ist ja auch irgendwo der Aufhänger
%           Der Satz, "tlb is on the critical path of everything really" sollte irgendwann mal kommen
%   - Motivation für vm
%   - Organisationsstrukturen auch im vergleich -> Fazit: Page Tables sind überall und werden tiefer -> vor allem wegeb backwards compatiblity?
%   - Hardware Strukturen für VM - MMU, TLB
%   - Operating system and VM -> Implemented by OS but fixed structures given by MMU
%   -> Problem, fehlende flexibilität ->  A look at several ...
%   -> source: Architectural and operating system support for virtual memory
%   source: issues of implementation

% -------------------------------------------------------------------------------------------------
% A little bit of history?? -> [Denning VM '96] -> Altas
% -------------------------------------------------------------------------------------------------

% -------------------------------------------------------------------------------------------------
%                                           VIRTUAL MEMORY
% -------------------------------------------------------------------------------------------------

This chapter introduces some essential concepts and mechanism which form the basis of the following
chapters. It first gives an overview of \textit{Virtual Memory}, its core requirements,
tradeoffs and implementations.\\
Then the hardware components and caches used to accelerate \textit[Virtual Memory] systems are
presented.\\
An overview of purely software-managed systems follows with a comparison of the general trade-offs
between software-managed and hardware-managed \textit{Virtual Memory} systems.\\
Finally, some specifics of the memory system of the chosen implementation platform, \textit{RISC-V},
are shown.

\section{Virtual Memory}
\textit{Virtual Memory} was first introduced in the Atlas System \cite{fotheringham1961dynamic} to
automate the task of swapping pages between main and secondary memory.
The idea was to make the programms completely unaware of of the real, physical memory by providing
an abstraction layer called virtual memory \cite{denning1996virtual}.\\
With virtual memory, programs appear to have the whole memory space of the machine at their disposal (flexibility) and
nobody to share it with (isolation).\\
The task of managing a programs memory, no matter where in its virtual address space it is, and putting it somewhere
in physical memory now falls to the operating system \cite{denning1970virtual}.\\
This was not only a nice-to-have, but a necessity. The size of programs was increasing faster than the size of main
memory did; while single programs
still fit in memory, operating systems allowed running multiple programs at once, collectively exceeding
the available physical memory \cite{tanenbaumOS}.
% -------------------------------------------------------------------------------------------------

\subsection{Virtual and physical addresses}
The terms of \textit{Virtual} and \textit{Physical} addresses will be coming up a lot in the course of this paper.
Virtual addresses refer to the addresses that are used by the program to reference memory object in its address
space. They are only valid in the programs own address space and may thus be reused by different programs.
Virtual addresses are translated to physical addresses, which are actual addresses to memory locations
on main memory. The \textit{Virtual Memory System} is tasked with performing this translation, creating mappings
and keeping book of those mappings.
\\
Some memory systems using an inverted page tables also use the notion of \textit{effective} addresses.
These form an addition layer of indirection that allows the sharing of pages \cite{jacob1998virtualissues}.\\
Effective addresses will not be discussed here any further.
% -------------------------------------------------------------------------------------------------

\subsection{Memory System Requirements}
In modern systems, \textit{Virtual Memory} does a lot more than swapping pages between main memory and secondary
storage. It is the foundation for a number of requirements to the memory system that are taken for granted
\cite{jacobSoftwaremanagedAddressTranslation1997}.\\
These requirements are as follows:

% -------------------------------------------------------------------------------------------------

\paragraph{Address Space Protection / Isolation} It should not be possible for one process to access the data
of another process, unless explicitly shared.
\cite{jacobVirtualMemoryContemporary1998}

% -------------------------------------------------------------------------------------------------

\paragraph{Shared Memory} Sharing memory allows programms to work on the same physical data with potentially
differing virtual addresses.
\cite{jacobVirtualMemoryContemporary1998}

\begin{marginfigure}
    \includegraphics*[width=1\marginparwidth]{figures/fund_share.pdf}
    \caption{\textbf{Page Sharing}}
\end{marginfigure}

% -------------------------------------------------------------------------------------------------

\paragraph{Large Address Spaces} Programs tend to require more and more memory. Swapping pages can help
to support programs bigger than the actual size of main memory \cite{tanenbaumOS}, but programs may
also require more memory than the theoretical limit set by hardware and software. Modern architectures
use bigger addresses to support programs that require a lot of memory \cite{jacobSoftwaremanagedAddressTranslation1997, jacobVirtualMemoryContemporary1998}.

\todo{pages already definied?}
% -------------------------------------------------------------------------------------------------

\paragraph{Fine-grained Protection} From a security perspective, it is often not desirable to allow the code segment
of a program to be writeable and the data section to be exectuable. \textit{Virtual Memory Systems} provide read-only,
read-write and exectue-only protection on a page granularity\cite{jacobSoftwaremanagedAddressTranslation1997}.
Illegal references may trigger exceptions, which allow the operating system to deal with the program \cite{jacobVirtualMemoryContemporary1998}.

\todo{more on tlb reach in HW section}
% -------------------------------------------------------------------------------------------------

\paragraph{Superpages}
Structures may be bigger than a single page and may thus occupy multiple entries of the translation caches while
only refering to one object. To avoid displacing other caches entries, \textit{Virtual Memory Systems} use
superpages to provide space for bigger objects to increase the reach of cache entries \cite{jacobSoftwaremanagedAddressTranslation1997}.

% -------------------------------------------------------------------------------------------------

\paragraph{Flexibility} Programmers should not have to think about the management
of the resources it requires to do its job, that is the ooperating systems job
\cite{tanenbaumOS}. As such, it should be possible to place the programs code
and data anywhere in the programs virtual address space to make the programmers
job as easy as possible
\cite{jacob1998virtualissues}. % Simplification of the programmes job

\begin{marginfigure}
    \includegraphics*[width=0.9\marginparwidth]{figures/fund_flexibility.pdf}
    \caption{\textbf{Flexibility} Program segments can be dispersed anywhere
        around the virtual address space; the Virtual Memory System has to place
        the pages into actual physical memory.}
\end{marginfigure}
% -------------------------------------------------------------------------------------------------

\paragraph{Sparsity} Big addresses and fine granularity result in a huge address
space with a sparse population for programs that do not require a lot of pages
\cite{tanenbaumOS}.

\begin{marginfigure}
    \centering
    \includegraphics*[width=0.5\marginparwidth]{figures/fund_sparsity.pdf}
    \caption{\textbf{Sparsity / Large Address Spaces} Virtual Memory Systems need to efficiently
        realize huge address spaces with only a few pages being used.}
\end{marginfigure}

% -------------------------------------------------------------------------------------------------

% \paragraph{Memory Mapped IO}
% \cite{tanenbaumOS}

Durch das immer weiter Auseinanderdriften der CPU und Memory geschwindigkeiten und vor allem
durch das größer werden von Addressräumen (von 32 auf 64 bit!) \todo{cite} sind die Ansprüche an
das Virtual Memory system gewachsen und so mussten sich auch die Implementationen weiterentwickeln.

% VM Properties and benefits: Was für Nutzen hat VM?
% Fazit: VM hat sehr nützliche Eigenschaften
% More: jacob1998virtualissues

% -------------------------------------------------------------------------------------------------
%                                IMPLEMENTATION OF VIRTUAL MEMORY
% -------------------------------------------------------------------------------------------------

% Wie können wir VM realisieren? -> Verschiedene Implementationen, Aber hier nicht auf die HW eingehen
%   [ A look at several...]
%   [ Issues of implementation]
\subsection{Implementation of Virtual Memory}
Every virtual memory system has to realize a mapping from virtual addresses of each processes
private, virtual address space to physical addresses that index data in main memory.
This section will provide an overview of how most commonly used virtual memory implementation.\\

% Naive approache
The naive approache is to simply have a big array in main memory. This array can be indexed
using the virtual addresses, at which the physical address to that virtual address is placed.\\
In a 32 bit address space with 4 KB pages, this requires 20 bits per page table entry. This
adds up to \[ 20 * 2^{20} bit = 20971520 / 8 Byte = 2.5 MB \].
To properly isolate the processes from each other, every process needs to have one of those arrays.
To realize fine-grained protection of pages, there would also need to be some bits per array entry
for read/write/execute rights.\\
With 64 bit computer, the space requirements for such page tables would be even higher.

% Hierarchical
\subsubsection{Hierarchical page tables} \todo{Different names -> from the survey papers
[a look at several issues of impl], Radix Tree}
To reduce the memory cost of managing pages, most \todo{cite?} virtual memory systems use a
multi-level page table, also known as a hierarchical page table. However, its structure is less like a
is less like a table and more like a tree, where the nodes are tables of page table entries (PTEs).
Here the virtual page number is divided into several parts. Each part of the VPN (Virtual Page Number)
is used to index a smaller table. The indexed PTE then points to the next smaller page table,
which in turn is indexed by the next part of the VPN. Depending on the implementation
this can involve up to 5 further indirections. A common scheme, as specified in the RISC-V ISA,
is Sv39, which provides a total of 3 levels per page table tree\cite{riscvreader}.
This scheme is also shown in the figure \ref{fig:fund}.

\begin{figure*}[t]
    \centering
    \includegraphics[scale=.8]{figures/VM-Tree.pdf}
    \caption[RISC-V Sv39 3-Level Page Tree]{Three-step page walk with a RISC-V Sv39 Page Table Tree:
        The value in the \texttt{satp} register is the base of the root page table; \texttt{VPN[2]}
        is the index into the root page table; the indexed \texttt{PTE} points to the next page table.
        This traversal continues until the bottom of the page table is reached. The last \texttt{PTE}
        contains the \texttt{PPN} of the physical address which can then be combined with the offset
        bits to make the full physical address}
    \label{fig:fund:pagetree}
\end{figure*}

% USE DEEPL WRITE FOR ALL FOLLOWING

% Conclusion -> Many main memory accesses -> Expensive
Traversing the page table to find the mapping requires an additional memory reference for each
level\todo{cite}, and since finding the mapping is on the critical path of every memory operation,
in the worst case, if all caches are missed, up to 5 memory accesses may be required (in a 5-level paging
scheme) just to find the mapping for a single memory access.

% Solution? Hashed!
\subsubsection{Inverted page tables}
An alternative paging scheme approaches the problem from the opposite direction and provides a PTE
for each physical frame instead of one entry per theoretical virtual page.
A physical page frame is a page-aligned space in physical memory for a page.
So the number of physical frames is determined by the size of main memory divided by the page size.
\todo{Pages already defined? Variable size too?}.
This has the enormous advantage, especially for 64-bit address spaces, that only as many pages need to be
as physically exist.
The page table design also has the advantage that, in the best case, significantly fewer main memory
accesses are required. In the simple design shown in figure \ref{fig:fund}, the corresponding
page table entry can be found with just two memory lookups \cite{skarlatos2020elastic}.

% yyWorst case -> Collision count limited only by page frame number
However, as the page frame indices are calculated by the VPN using a hash function, collisions may occur,
hash collisions can occur. Since the length of a collision chain is unpredictable, the maximum number of memory
of memory accesses required to find the correct PTE is only limited by the maximum number of
page frames and therefore by the size of the main memory.

% Hierarchical -> fixed number of refs (but memory usage)
Hierarchical multi-level page tables have the key advantage that they always require
a fixed number of memory accesses to determine the PTE.

% Hash anchor to reduce the number of collisions
Typical inverted page table designs often include a so-called hash anchor table, which is placed between the output of the
between the output of the hash function and the page table. If the hash anchor is twice the size of the
of the page table, the average collision chain length can be halved.
However, at least one additional memory reference is required in any case \cite{jacob1998virtualissues}.
Alternative designs for inverted page tables allow the table to be dynamically resized to avoid hash collisions.
hash collisions. However, this is very expensive and is generally avoided \cite{skarlatos2020elastic}.

% Inverted Page Table figure
\begin{figure*}[t]
    \centering
    \includegraphics[scale=1]{figures/inverted_pt.pdf}
    \caption[Simple Inverted Page Table Design]{A inverted page table has an entry for every physical
        page frame, reducing memory accesses to a minimum of one. Collisions in the hash table (red arrows) can
        make the access much more expensive. }
    \label{fig:fund:inverted}
\end{figure*}

% Conclusion -> Disadvantages of inverted page tables [see also hash don’t cache!]
Inverted page tables significantly reduce the average access time to the PTEs, but they make it
more difficult to support other features like superpages and memory sharing.
The Power Architecture, for example, supports this through a two-stage translation process \cite{yaniv2016hash}.
% [hash dont cache] -> Hashed paging performs better


% [hash dont cache] -> superpages are harder

% [issues of impl] ->

% [ A look at several] ->

% -> Most commonly used in today's hardware -> Multi level page tables
There is no clear winner in the debate between hashed vs. radix \todo{still need to introduce the term "radix"}
and commercial hardware supports a variety of designs that are not standardized and, in some cases,
differ significantly \cite{jacob1998look}.


Modern Intel processors now support radix designs with a depth of up to 5 levels,
which still use 4KB pages to maintain compatibility.

% Todo, aber größere pages erhöhen tlb reach usw
% Fazit -> Hauptproblem von VM sind teure Hauptspeicherzugriffe im kritischen Pfade von allen Memory Operations

\todo{discussion: Sind die aktuellen VM systeme (vor allem hierachical) noch Zeitgemäß oder nur noch altlast???}
% -------------------------------------------------------------------------------------------------
%                                  END SECTION - VM IMPLEMENTATION
% -------------------------------------------------------------------------------------------------



% -------------------------------------------------------------------------------------------------
%                                            VM HARDWARE
% -------------------------------------------------------------------------------------------------
\cite{denning1970virtual} % Hardware support source
\section{Memory Management Hardware}
To further accelerate the translation of virtual to physical addresses, most modern computers use additional
hardware components. These consist of a hardware page table walker (MMU) and a translation cache,
commonly referred to as a Translation Lookaside Buffer (TLB) \cite{jacobVirtualMemoryContemporary1998}.

% Frage: Besteht das MMU aus TLB + State Walker oder ist MMU der State Walker und TLB einfach extra?
% FIGURE Simple HW Architecture for VM Acceleration Hardware
\begin{figure*}[t]
    \centering
    \includegraphics[scale=1.2]{figures/simple_mmu_arch.pdf}
    \caption[A simplified architecture of CPU, MMU and TLB]{A simplified architecture of CPU, MMU and TLB:
        User-level programs running on the \texttt{CPU} try to access main memory with virtual
        addresses; virtual addresses get transparently translated to physical addresses by the
        \texttt{MMU} by either looking up the address in the TLB or by performing a page table
        lookup with the hardware-supported page table design}
    \label{fig:fund:simplearch}
\end{figure*}
% End Figure

% -------------------------------------------------------------------------------------------------

% Davor sollte der Page Table walk bekannt sein
\subsection{MMU}
The Memory Management Unit (MMU) takes on the task of address translation for the computer.
It sits between the processor, which primarily works with virtual addresses, and the memory bus,
which is accessed with physical addresses. When the processor accesses a certain virtual address,
the MMU performs a page table walk to determine the physical address corresponding to the virtual
address. During this process, the processor is effectively frozen \cite{jacobVirtualMemoryContemporary1998}
\todo{more fine-grained citation? -> Probably not, these are only fundamentals, should however quote overarching works}

\subsection{TLB}
Since it would be very costly to traverse the page table in hardware for every load or store memory access,
there is a cache for the translations, the Translation Lookaside Buffer (TLB). This cache contains
the most recent translations from virtual to physical addresses.\\
The MMU can first check the TLB, which can be searched in parallel and thus extremely quickly \cite{drepper2007every}.



% Gute quelle für alles hier : [jacob2010memory]
\cite{jacobVirtualMemoryContemporary1998}
% \subsection{A typical Page Table Walk}

% -------------------------------------------------------------------------------------------------
\subsection{PWCs}
\cite{barrTranslationCachingSkip}
\cite{yaniv2016hash}

% -------------------------------------------------------------------------------------------------
\subsection{Address Space Identifies}
% \cite{jacobSoftwaremanagedAddressTranslation1997}
%TODO in the vm requirements , ASIDs are presented as HW-support for address space protection
With Address Space Identifiers (ASIDs), RISC-V provides a way to more efficiently utilize virtual memory:
Since every process has its own virtual address space, translations that are still present in the
TLB may not be valid after a context switch.\\

\textbf{SFENCE.VMA} RISC-V provides one instruction that acts primarily as a memory barrier:
It prevents reordering of instructions accessing memory accross the \texttt{sfence.vma} instruction.\\
This is important when the page tables are switched, e.g. when the kernel is entered from user mode,
because the translations will change.\\
The instruction also acts as a flushing operation for TLB entries. With both of the optional registers
arguments set to zero, the instruction will flush all entries.\\
Setting the first register to a specific address will only flush translations containing that address.\\
The second register specifies the address space that should be flushed, given a ASID.

This mechanism allows for a precise control over flushing of TLB entries, which can improve the
memory system performance \cite{RISCVInstructionSet}.



% -------------------------------------------------------------------------------------------------


% Fazit -> Es gibt hardware strukturen die VM beschleunigen können, die machen es auch schneller
%       ABER: Die machen die VM Software Systeme auch sehr viel rigider und unflexibler
% Kurze Diskussion -> Machen Hardware strukture das wirklich schneller? [ A look at ...]



% -------------------------------------------------------------------------------------------------

% -------------------------------------------------------------------------------------------------

\section{Sofware-based Virtual Memory System}
% VM in Software möglich mit ähnlicher Performance wie hw möglich -> Sollte ja mehr Flexibilität geben
%   [ A look at several...]
% Software managed address translation
% \subsection{Guarded Page Tables}






% -------------------------------------------------------------------------------------------------
% -------------------------------------------------------------------------------------------------
\section{HW VM vs SW VM}
% related work will then come in to discuss approaches close to my approach
% With only software ptw process would have to context switch to the kernel -> Very expensive
% With an MMU the processor essentially just freezes until the memory operation has completed
% \subsection{HW-Dependent PTE Structure} -> inflexibility
There are several considerations that should be taken into account when comparing hardware and
software-managed translations.

\paragraph{Fixed Paging Structures} In hardware-managed virtual memory, the structures for page tables
and page table entries are fixed by the microarchitecture. As a result, the operating system cannot tailor
memory management to its purposes and use case, and it is stuck with the fixed design.
This also complicates the portability of system software, as there is no standard for these memory management
structures. Despite there being no significant performance differences among the various designs,
there is no standardization \cite{jacob1998look}.

\paragraph{Pipeline Freezing / Flushing} On a TLB miss \todo{already defined?},
with a hardware-managed TLB, only the pipeline freezes (at least for instructions dependent on
the memory access). However, with a software-managed TLB, control is handed back to the operating system
via an exception, which notes that the required address is not in the TLB (TLB miss exception).
The jump back to the operating system causes a context switch, requiring the state of the current process
to be saved. During this, the reorder buffer is flushed, and the pipeline is heavily disrupted.
Switching to the kernel can also lead to further data and instruction misses in the long term,
as the kernel entry likely overwrites some cache lines that the running process needed. \todo{Cite}

\paragraph{Embedded} Using virtual memory is becoming increasingly relevant for embedded systems.
They would certainly benefit from more flexible designs. Additionally, saving on chip size by
eliminating the MMU would also reduce production costs \cite{jacob1998look}. \todo{not sure about this}



% Conclusion of HW vs SW
\cite{jacob1998look} concludes in a comparative study of various HW and SW memory designs
that hardware-based approaches are generally more performant, but software-based designs are certainly
viable if the caches are large enough to reduce the number of cache misses.
Especially in terms of flexibility, software-based approaches have a significant advantage,
as the VM system can be fully defined by the operating system.

% -------------------------------------------------------------------------------------------------

% -------------------------------------------------------------------------------------------------

% TODO Short discussion hw and sw vm -> common problem: Page Table Walks require in either case a
% lot of memory references. These costs can be aleviated using caches, but will still cost [ cite a source abouts costs here]

\todo{section on basic operating system structures like exception handler, load store sandwich}

\todo{fundamentals on the implementation platform}
% -------------------------------------------------------------------------------------------------

% -------------------------------------------------------------------------------------------------


\section{RISC-V Basics}
The implementation presented in this work runs on a RISC-V platform. Therefore, it is necessary
to go over some basic concepts of the RISC-V platform. Particularly relevant here are the virtual
memory system, the exception/trap mechanism, and the control and status registers (CSRs), which
form the foundation for extending RISC-V.

% -------------------------------------------------------------------------------------------------

\subsection{Sv39 Virtual Memory}
\label{fund:sv39}

\todo{explain why the top bits in PTE are all zero (with leichten Bezug zu xv6 impl)}
\begin{figure*}[h!]
    \centering
    \begin{bytefield}[bitwidth=\widefigurewidth/64,bitheight=\widthof{~PBMT~}, bitformatting={\tiny\bfseries}, boxformatting={\centering}]{64}
        \bitheader[endianness=big]{63,62,61,60,54,53,28,27,19,18,10,9,8,7,6,5,4,3,2,1,0} \\
        \bitbox{1}{N} &
        \bitbox{2}{\rotatebox{90}{PBMT}} &
        \bitbox{7}{Reserved} &
        \bitbox{26}{PPN[2]} &
        \bitbox{9}{PPN[1]} &
        \bitbox{9}{PPN[0]} &
        \bitbox{2}{\rotatebox{90}{RSW}} &
        \bitbox{1}{D} &
        \bitbox{1}{A} &
        \bitbox{1}{G} &
        \bitbox{1}{U} &
        \bitbox{1}{X} &
        \bitbox{1}{W} &
        \bitbox{1}{R} &
        \bitbox{1}{V}
    \end{bytefield}
    \caption[RISC-V Sv39 Page Table Entry]{RISC-V Sv39 Page Table Entry}
    \label{fig:theory:sv39pte}
\end{figure*}
% -------------------------------------------------------------------------------------------------

\subsection{Traps}
\todo{Should this be in fundamentals?}
% exceptions vs interrupts
Traps are part of the RISC-V privileged architecture. They provide a mechanism to respond to
external events and unusual runtime events, known as exceptions \cite{riscvreader}.
The term "trap" is an umbrella term that is further divided into interrupts — asynchronous events —
and exceptions — synchronous events.\\
Exceptions are particularly of interest here, as a TLB miss occurs during the execution of an instruction,
meaning it happens synchronously with the processor's clock. However, it is also important to keep interrupts
in mind when working with Qemu and xv6 source code, because on one hand, the Qemu code handles exceptions
and interrupts within the same functions, and on the other hand, xv6, or RISC-V in general, uses a unified
vector for handling both interrupts and exceptions \cite{RISCVInstructionSet}.\\

% Exception registers
There are six central registers for triggering and handling exceptions. These exist both for
Supervisor Mode (prefixed with "s") and Machine Mode (prefixed with "m").
\todo{Have privilege modes been introduced already?}
Whether the machine mode or supervisor mode version of the register should be used for an exception depends
on the mode in which the exception is handled. In the following, all registers are presented with the
M-Mode prefix only \todo{Has M-Mode as an abbreviation for Machine Mode been introduced?}.
\todo{Because the exception was ultimately designed as an M-Mode exception?}

% TODO: sorting of registers by role?

% Vector -> mtvec
\textbf{Exception Vector} The hart \todo{Has the term "hart" been introduced?} experiencing
an exceptional state must know where the kernel routine is located to handle the exception.
The \texttt{BASE} field of the register contains a 4-byte aligned physical address to which
the program counter is set in the case of an exception.\\
The \texttt{MODE} field allows switching between \texttt{direct} and \texttt{vectored} modes.
In \texttt{MODE=Direct}, the PC is set to BASE for all traps, whereas in \texttt{MODE=Vectored},
the PC is set to $BASE+4*CAUSE$ for asynchronous interrupts.\\

% Delegation -> medeleg

% Data for exception handling -> mcause, mtval, mepc, mscratch
\textbf{Context Information} To properly handle the exception, some context information is required.
The \textbf{mcause} register contains the exception code of the exception; \textbf{mepc} contains
the program counter of the instruction that triggered the exception; and \textbf{mtval} holds
exception-specific information, such as the virtual address that triggered a page fault exception.
\textbf{mstatus} contains general information about the current hardware state.\\

\textbf{Delegation} Normally, all exceptions are handled in machine mode; however, in some cases,
it may be useful to handle the exception in a lower privilege mode. With the bitfield in the
\textbf{medeleg} register, individual exceptions can be chosen to be delegated to the next-lower
privilege mode.\\

% Exception number
\textbf{Exception Code} Each exception is assigned a unique number, the exception code \cite{riscvreader}.
This can be found in the \texttt{mcause} register when handling the exception.


% Zusammenspiel der Register im Exception Fall -> xv6 Book Exception Machinery

\textbf{What the Hardware does} Wenn eine Exception getriggert wird macht die Hardware folgendes:
\begin{enumerate}
    \item Interrupts are disabled by clearing the MIE bit in \textbf{mstatus}
    \item PC is copied to mepc
    \item the current mode is saved to the MPP field in mstatus
    \item mcause is set to the proper exception code
    \item the mode is set to the machine mode
    \item Pc is set to stvec
    \item execution continues at the new pc
\end{enumerate}
% -------------------------------------------------------------------------------------------------

\subsection{Contol and Status Registers}
% Section describing RISC-V CSRs -> Originally in theory
\todo{table of csr space in appendix?}
The RISC-V ISA provides a 12-bit encoding space for 4096 CSRs. A CSR address is logically split
into four parts: The top two bits \texttt{csr[11:10]} specify whether the CSR is read/write or read-only;
\texttt{csr[9:8]} encode the minimum priviledge level that is allowed to access the CSR; \texttt{csr[7:4]}
may be partially used to define a specific use for a range of CSRs. E.g. CSRs with an address
between \texttt{0x7B0} and \texttt{0x7BF} shall be used for Debug-mode-only CSRs. The format
of the CSR addresses is also depicted in figure \ref{fig:theory:csr}.

% RISC-V CSR address bytefield
\begin{figure*}[h!]
    \centering
    \begin{bytefield}[bitwidth={2em}, bitformatting={\bfseries}, boxformatting={\centering}]{12}
        \bitheader[endianness=big]{11,10,9,8,7,4,3,0} \\
        \bitbox{2}{RW/RO} &
        \bitbox{2}{Priv} &
        \bitbox{4}{Usage} &
        \bitbox{4}{Index}
    \end{bytefield}
    \caption[RISC-V CSR address format]{RISC-V CSR address format}
    \label{fig:theory:csr}
\end{figure*}

Each legal CSR address identifies a CSR. The size of the CSRs identified by the CSR address depends
on the values of the SXLEN and UXLEN fields in the \textbf{mstatus} register. Currently, the
specification \cite{RISCVInstructionSet} allows for 32 bit, 64 bit and 128 bit.\\ \todo{disclaimer which we will be using here?}
% Short elaboration on using the csr to write TLB entries -> in the end its on the hardware implementor
RISC-V provides dedicated instructions for read/write and bit manipulation with both register values
and immediates.
\todo{ hardware overhead for writing TLB with CSRs -> future work, is it worth it??}

% What format, how big, how many csrs?
Now with CSRs we have a mechanism to add custom behaviour to the ISA. That answers the ''How?''
of writing TLB entries in software.\\
The next step is to figure out what data is required to create a TLB entry and in what format
this data is best communicated via the CSRs to the computer.\\
% -------------------------------------------------------------------------------------------------






% All approaches are based on a table.
% Überleitung zu meinem Thema -> Avoid all memory references and just have a simple (hash?) function that realizes VM
% [ a look at severeal ] -> conlsion


\begin{figure*}[t]
    \centering
    \begin{bytefield}[bitwidth=\widefigurewidth/56,bitheight=\widthof{~PBMT~}, bitformatting={\tiny\bfseries}, boxformatting={\centering}]{56}
        \bitheader[endianness=big]{55,30,29,21,20,12,11,0} \\
        \bitbox{26}{PPN[2]} &
        \bitbox{9}{PPN[1]} &
        \bitbox{9}{PPN[0]} &
        \bitbox{12}{Page Offset}\\
    \end{bytefield}
    \caption[RISC-V Sv39 Physical Address]{RISC-V Sv39 Physical Address}
    \label{fig:theory:sv39pa}
\end{figure*}


\begin{figure*}[t]
    \centering
    \begin{bytefield}[bitwidth=\widefigurewidth/64,bitheight=\widthof{~PBMT~}, bitformatting={\tiny\bfseries}, boxformatting={\centering}]{64}
        \bitheader[endianness=big]{63,60,59,44,43,0} \\
        \bitbox{4}{Mode} &
        \bitbox{16}{ASID} &
        \bitbox{44}{PPN} \\
    \end{bytefield}
    \caption[RISC-V Sv39 \texttt{satp} CSR]{RISC-V Sv39 \texttt{satp} CSR}
    \label{fig:theory:sv39satp}
\end{figure*}

\begin{figure*}[t]
    \centering
    \begin{bytefield}[bitwidth=\widefigurewidth/39,bitheight=\widthof{~PBMT~}, bitformatting={\tiny\bfseries}, boxformatting={\centering}]{39}
        \bitheader[endianness=big]{38,30,29,21,20,12,11,0} \\
        \bitbox{9}{VPN[2]} &
        \bitbox{9}{VPN[1]} &
        \bitbox{9}{VPN[0]} &
        \bitbox{12}{Page Offset}\\
    \end{bytefield}
    \caption[RISC-V Sv39 Virtual Address]{RISC-V Sv39 Virtual Address}
    \label{fig:theory:sv39va}
\end{figure*}

\chapter{Related Work}

\label{chap:related}
This work can be broadly classified to be in the field of Virtual Memory optimization. There is a large body of literature that deals with optimizing the VM system, as it lies on the critical path of every memory operation.

There are different approaches or perspectives to addressing the system (this is not an exhaustive list):

\begin{itemize}
    \item page table structures and their optimization, e.g., inverted or hierarchical
    \item caching of pages, PTEs in the TLB, and cache indexing
    \item cache replacement strategies
    \item cache sizes, associativity of caches
    \item Page Walk Caches (PWCs) to store partial translation results
\end{itemize}


% -------------------------------------------------------------------------------------------------

% Guarded Page Tables
\textbf{\cite{liedtkeGPT}} shows an innovative design leveraging the flexibility of software-managed
address translation.
The design of Guarded Page Tables (GPTs) is based on hierarchical page tables, but allows skipping over levels of the table to reach translation results faster.
This proves to be very effective for increasing the performance of Virtual Memory systems, as the biggest penalty comes from page tables walks needing to reference memory for every level in the page table.
Gernot Heiser showcases a practical implementation of GPTs in the L4/MIPS system \cite{heiserAnatomyHighPerformanceMicrokernel}.



% -------------------------------------------------------------------------------------------------

% In Cache Software Managed Address Translation
\textbf{\cite{jacobSoftwaremanagedAddressTranslation1997}} explores software-managed address translation and analyses the efficiency of a PowerPC implementation
of the presented design they call \textit{softvm}. They show that
software-managed address translation can achieve better performance and
at the same time simplify hardware by dispensing with translation caches and
the hardware state-machine for walking the page table.
The approach is based on handling virtually indexed and tagged cache misses in
software.
With sufficiently sized virtual caches the system can go for long periods
without requiring translations.
Not unlike this paper, they extend the PowerPC architecture by two new instructions
to write entries to the cache. However, their design uses software page table
walks to find translations on cache-miss, while the approach presented in this
paper presents a software-based TLB fill mechanism with segmented
memory allocation.

% -------------------------------------------------------------------------------------------------

% Translation Caching skip dont walk -> Translation Caches
\textbf{\cite{barrTranslationCachingSkip}} examines the design space of translation caches in MMUs. These caches are not the same as TLBs: TLBs contain full translation results, while the translation caches considered here contain partial translations. In the case of a cache hit, these partial translations allow the system to skip individual steps when traversing the page table tree, thereby saving one or more memory accesses. Otherwise, each level of translation requires a memory reference. These caches are also referred to as Page Walk Caches (PWC) \cite{yaniv2016hash}.

The specific translation cache designs of AMD and Intel platforms are examined and compared to three other designs proposed by the authors. Barr et al. conclude that radix page tables, by caching entries at higher page table levels, can outperform inverted page table designs.

PWCs are out of scope for this thesis, but might be examined in future work with the mapping function approach. This work focuses on optimizing the memory path by eliminating page tables and using a software-controlled TLB, and it does not further consider caches aside from the TLB.



% -------------------------------------------------------------------------------------------------

% Hash don't Cache
\textbf{\cite{yaniv2016hash}} challenges the results of \cite{barrTranslationCachingSkip} and argues that the obtained results are based on a suboptimal implementation of the inverted page table.

They conclude that a well-optimized inverted page table can outperform a radix page table equipped with PWCs. However, they also address the conceptual disadvantages of inverted page tables. For example, it is more difficult to implement superpages.

The work takes a closer look at the differences between various page table designs and the requirements of a memory system (such as superpages or page sharing). These requirements are also important for the design presented here. However, this work aims to avoid using page table structures altogether.


% -------------------------------------------------------------------------------------------------

% Every walk’s a hit: making page walks single-access cache hits
\textbf{\cite{park2022every}} identifies that today’s memory capacities far exceed the coverage of TLBs, causing memory-hungry applications to suffer from frequent page table walks (PTWs).

Two approaches are presented to reduce the associated costs: The first approach aims to reduce the number of memory references per PTW by combining two levels of the page table into one. The second approach modifies the cache replacement policy so that cache entries containing PTEs are more likely to remain in the cache during periods with many TLB misses, allowing PTWs to run directly from the cache instead of being loaded from main memory.

They show a 2.3\% performance improvement from flattening the page table tree, 6.2\% through cache prioritization, and a combined performance improvement of 9.2\%. Both approaches focus on optimizing access to page table structures and are separate from this work, as this work aims to eliminate these structures entirely.


% -------------------------------------------------------------------------------------------------

% Elastic Cuckoo Page Table
\textbf{\cite{skarlatos2020elastic}} presents a novel page table design called \textit{Elastic Cuckoo Page Tables}. The design exploits memory-level parallelism to enable fully parallel page table lookups. At the core of the design is the Elastic Cuckoo Hashing algorithm, which allows multiple hashing locations for a given element and enables efficient, gradual resizing of the hash table. Skarlatos et al. demonstrate an application execution speedup of 3-18\% using the Elastic Cuckoo Page Table design.



% -------------------------------------------------------------------------------------------------

\cite{zagieboylo2020cost} proposes an approach that shifts responsibilities of memory management in
parts back to the applications: All applications get a fixed-size chunks of physical memory and then have to manage allocations across these blocks.
Thus, the task of memory management with the chunks falls to compilers, language runtimes and the applications themselves.
This approach does not require any address translation and thus gets completely rid of the overhead associated with the VM system.
Common features of Virtual Memory, like memory space protection can still be implemented with physical memory protection mechanisms present on commodity hardware.
Overall, this approach trades a reduction for complexity of the hardware with increased complexity on the software side.

The design presented in this paper resorts to bigger segments per process, but generally aims to keep the memory management responsibilities with the operating system.

% -------------------------------------------------------------------------------------------------

\cite{halbuer2023morsels} argues that prevailing memory management system designs are becoming increasingly inefficient given modern systems that have very big main memories and workloads that require large in-memory data sets.
The paper presents a novel approach addressing the limitations of current methods with \emph{Morsels}.
These Morsels are self-contained memory objects, spanning entire page table sub-trees, thus allowing efficient remapping, sharing and reducing memory overhead.
They reuse existing interfaces and work on top of existing page table structures to provide a supplementary layer to improve the efficiency of memory-intensive applications.

% -------------------------------------------------------------------------------------------------

The related work presented here focuses on optimizing current page table designs and their hardware
support. This work will explore the feasibility of implementing Virtual Memory without any page table whatsoever using a specialized mapping function to generate virtual to physical mappings.
This promises to reduce the overhead of TLB misses caused by expensive main memory access.


\chapter{Theory}
\label{chap:theory}

% Überblick über die Idee schaffen -> Virtueller Speicher durch funktion möglich

% -------------------------------------------------------------------------------------------------
%                                            Idea
% -------------------------------------------------------------------------------------------------

\section{The Idea}
% Idea Description

% Steps necessary to implement Idea

% Gegenüberstellung: Normaler HW Page Table Walk - Mapping function

% Normal Page Table Walk
\begin{figure*}[ht!]
    \centering
    \includegraphics[scale=1.5]{figures/theory_normal_tlb_miss.pdf}
    \caption[Usual TLB Miss]{This figure shows what usually happens when the TLB misses:
        the miss will invoke the hardware state machine page table tree walker; the walker traverses
        the page table tree and if a valid PTE is found, the mapping is added to the TLB. The processor
        then executes the failing instruction again which will then result in a TLB hit}
    \label{fig:theory:normal_tlb_miss}
\end{figure*}

% TODO der Zwischenschritt mit dem Software Page Table walk wird dann in der Impl
% eingeführt

% Virtual Memory using a Mapping Function
\begin{figure*}[ht!]
    \centering
    \includegraphics[scale=1.5]{figures/theory_mapping_fx.pdf}
    \caption[Virtual Memory using a mapping function]{Instead of emulating a hardware page table walk in software on
        a TLB Miss exception, a mapping function is invoked that calculates the PTE using arithmetic primitives and tries
        to avoid memory references.}
    \label{fig:theory:mapping_fx}
\end{figure*}





% The Idea -> Getting rid of any memory access in favor of simple arithmetic operation (eg. hash functions)
Die grundlegende Idee dieser Arbeit ist es die Page Table Strukturen, seien es Hashed Page Tables oder
Radix Page Tables, oder andere Designs wie sie in Related work presentiert wurden, komplett loszuwerden.
Diese sollen durch eine einfache funktion ersetzt werden die das Mapping von virtuellen zu physischen
Addressen realisiert.\\
Offensichtliche Kandidaten für so eine Funktion wären natürlich Hashfunktionen. Vor allem die in
\cite{skarlatos2020elastic} vorgestellte Elastic Cuckoo Hash function verfügt schon über Eigenschaften
die für ein direktes virtual-physical mapping, statt einem virtual-hash table index, nützlich wären.
\todo{Das mit den Cuckoo hash ist vielleicht für diese Stelle schon zu viel detail}.
Der wichtigste Aspekt diese Funktion ist allerdings, dass sie, anders als alle anderen Page Table designs,
keine Speicherorte referenziert, also praktisch komplett ohne einen Zustand auskommt.\\
Die phyische Addresse soll also möglichs nur aus informationen,  die in registern zur verfügung
gestellt werden können, ermittelt werden.\\
% Wie related das zu: Demand Paging?, Virtual Memory Requirements, Features wie page sharing
Für einen ersten Proof of Concept soll die funktion in einem Exception Handler implementiert werden.
Diese Excepion handler wird getriggert sobald es zu einem TLB miss kommt, das heißt wenn ein
gesuchtes Mapping für einen Speicherzugriff nicht im TLB zu finden war.

%elaboration on general costs of page table walks -> basically exactly what related
% work motivated their designs with

% Previous approaches aimed to reduce the memory footprint or optimize cache accesses, or reduce page table
% walks to a minimum, this approach aims to eliminate paging structures completely
While previous work tried to either reduce the number of memory accesses \todo{CITE Lidtke ,Skip no walk, single hit uppsala paper},
decrease the occurrence of a TLB miss and decrease the latency of handling a TLB Miss, this paper
presents an approach to create a virtual memory scheme not requiring paging structures at all.\\

Not unlike the software-managed address translation design presented by Jacob and Mudge \cite{jacobSoftwaremanagedAddressTranslation1997},
the \textit{softtlb} approach utilizes a exception triggered by a cache miss to invoke a software-defined
exception handler.\\
Unlike the \textit{softvm} design, this approach will be based on handling an exception that is triggered
when die TLB misses \todo{TLB and its role explained?}.
The handler code for this \textit{TLB\_Miss} exception will from now on be called \textit{TLB Manager}.

The \textit{TLB Manager}s job is then to resolve the exception by filling the TLB for the failing address
that triggered the exception.


%Implementation Platform
Diese Kapitel betrachtet zwar nur die Theorie des softtlb desings, allerdings muss diese in Kontext
der gewählten Platform gestellt werden und wird daher an manchen Stellen in tieferes Detail der konkreten
Platform eintauchen als man für eine theoretische Ausarbeitung des Designs erwarten würde.
Das ist wichtig um dem späteren Implementierungskapitel den nötigen Kontext zu liefern.\\
Als Platform für die Implementierung wurde der Einfachkeit halber das xv6-riscv educational operating
system \cite{cox2011xv6} gewählt.\\
\textit{Xv6} wird am MIT genutzt um Operating Systems Kurse zu bieten und bietet mit dem Handbuch eine
einfache und verständliche Platform die ein vereinfachtes Unix Version 6 Interface implementiert.\\
Xv6 gibt es in einer x86 version und in einer RISC-V version, es wurde hier die RISC-V version verwendet.
\\
Da im Rahme der Implementierung einer neuen Exception das nutzen echter RISC-V Hardware nicht möglich
ist, muss ein Emulator benutzt werden um die nötigen Änderungen an der ISA zu implementieren und zu nutzen.
Dafür wurde der Qemu Emulator gewählt, da dieser sehr Umfassend und performant ist und außerdem einen
TLB emuliert.
% Zukünftige Idee -> Configurierbarer "stateless" physical frame calculator hw baustein

% What to do to realize this idea? (basically top level overview of the theory and impl chapters)
Dieser Abschnitt enthält die theoretischen Ausarbeitungen zu den einzelnen Bestandteilen die für
die Implementation des Software kontrollierten TLB Miss Handlers benötigt werden. Diese Bestandteile
sind wie folgt:
\begin{itemize}
    \item Die Erweiterung des Qemu Emulators um einen Exceptionwurf bei TLB Miss
    \item Einen Maschine-Mode Traphandler in xv6 der die neue TLB Miss Exception entsprechend behandelt
    \item Eine Möglichkeit um TLB einträge mittels spezieller Instruktionen zu schreiben
\end{itemize}
% New Exception - TLB Miss Exception
% CSRs to write TLBs
% Exception Handler
% Exception triggerer for testing



\todo{elaboration on xv6 here or in fundamentals (or an extra platform chapter)?? -> because theory
    may be independent from the platform in big parts, but still needs to be taken into consideration (like with
    csrs)}

% -------------------------------------------------------------------------------------------------
%                                            Required Changes
% -------------------------------------------------------------------------------------------------
% Conclusion on previous Work -> Still need a memory access

% The Idea -> Getting rid of any memory access in favor of simple arithmetic operation (eg. hash functions)
% statefull -> stateless
% What to do to realize this idea? (basically top level overview of the theory and impl chapters)
% New Exception - TLB Miss Exception
% CSRs to write TLBs
% Exception Handler
% Exception triggerer for testing


% -------------------------------------------------------------------------------------------------
%                                            New Exception
% -------------------------------------------------------------------------------------------------

\section{New Exception}
% Allgemeines über Riscv Exceptions
\todo{ist dieser Abschnitt vielleicht eher der Idee zuzuordnen? So in der Art sollte es
    auf jeden Fall in die Idee, aber vielleicht nicht in diesem Detail, vielleicht nur einmal
    den ''typischen'' controlflow durchgehen und dann einmal den Modifizierten für softtlb}
\\
Damit das Betriebssystem im Falle eines TLB Misses sich um das füllen des TLBs kümmern kann
braucht es zunächst einmal einen Mechanismus um das Betriebssystem wissen zu lassen,
das es zu einem entsprechenden TLB Miss gekommen ist. In Betriebssystemen, die keinen
software managed TLB, wie z.b. MIPS \cite{MIPSArchitectureProgrammers2016} umfassen, ist der
Zugriff auf den TLB in der Regel komplett transparent für das Betriebssystem und somit auch wenn
es zu einem TLB miss kommt. Normalerweiße würde dann die Hardware State Machine aktiviert werden
die dann einen entsprechenden Page Table Walk durchführt um den TLB zu füllen.
\todo{
    explain normal pipeline for TLB misses -> like here + with repeating of fauling instruction
    -> tlb as interface between virtual and physical memory\\
    Aber vielleicht nicht hier
}\\
Statt das nun aber das MMU anfängt nach dem passenden PTE zu der virtuellen Adresse zu suchen,
soll ein Exception geworfen werden, die das Betriebssystem abfängt um den TLB dann in Software
zu füllen.\\
Im folgenden wird zunächst der allgemeine RISCV-Trap mechanismus betrachtet und dann
wird erleutert wie eine neue Exception in RISC-V hinzugefügt werden kann, bzw. was alles beachtet
werden muss.


% -------------------------------------------------------------------------------------------------
%                                            Exception Handling
% -------------------------------------------------------------------------------------------------
% Catching New Exception
%   Current xv6 machine mode exception handler ( -> maybe too implementation specific)
%   riscv interrupt mechanism (precice, vectorized)
%   Exception Catching theory, context switch, state save restore
\section{TLB Miss exception handling}
% Allgemeines über Exception Handling, Context Switches, etc
\subsection{xv6 exception handling}

% -------------------------------------------------------------------------------------------------
%                                            TLB Writing
% -------------------------------------------------------------------------------------------------
% Writing TLBs -> Implementation Theory
%       Starting with MIPS for inspiration on how instructions for TLB manipulation may look like
%       Then continue with a possible RISCV Implementation Idea
%       I-TLB and D-TLB out of scope
%
\section{TLB modification}

% -------------------------------------------------------------------------------------------------
%   Comparison to MIPS
%       MIPS TLB instructions
%       MIPS TLB entries -> Qemu TLB?
%       Replacement
\subsection{MIPS TLB} % ---------------------------------------------------------------------------

% Inspiration: MIPS
MIPS is a good source of inspiration for a possible CSR format design, since it already provides
instructions to modify the TLB.\\
The MIPS64 instruction set manual \cite{MIPSArchitectureProgrammers2016}
shows a number of different instructions concerned with the invalidation, probing, flushing, reading
and writing (indexed and random).\\
The most interesting for a first design would be the \texttt{TLBWR} instruction for writing a TLB
entry at a random index. With a similar instruction in RISC-V, we can already implement a purely
software-controlled virtual memory system.\\
The other types of TLB instructions that MIPS provides are not strictly necesarry,
except for flushing. Without being able to flush existing translations from the TLB,
user mode processes may try to access physical mappings stemming from other processes.\todo{can this even happen? -> why else would we need to flush the tlb??}
But the RISC-V priviledged Architecture already provides this functionality
with the \texttt{sfence.vma} instruction \cite{riscvreader}.

\paragraph{MIPS - TLBWR} The arguments for the instruction need to be written in some
other registers - \texttt{EntryHi}, \texttt{EntryLo0}, \texttt{EntryLo1} and \texttt{PageMask}.

% \section{L4/MIPS als Vorlage für Software TLB Management in RISCV (Qemu)}
% \subsection{TLB Miss Exception}
% \subsection{TLB Loading instructions}
% \paragraph{tlbwi}
% \paragraph{tlbwr}
% Comparison to MIPS tlb software


% Compare my implementation to a mips one -> Heiser TLB Fastpath
% Fixed TLB format, but flexibilty of CSR format

% Qemu TLB

% Typical TLB entry format

% selective TLB flushing -> RISCV ASIDs (or even no TLB flushing  because VAs contain ASID?)

% How is the RISCV Tlb indexed -> Check Qemu source

% Replacement policy
%   MIPS -> Indexed writes
A advantage of software-managed TLBs is, that the operating system can implement custom
TLB replacement policies, that may even change depending on workload, programs running
and other circumstances.\\
The default replacement strategy for the MIPS \texttt{tlbwr} instruction is to simply
use the value of the \texttt{C0\_RANDOM} register as the index for the next TLB entry
to be replaced. The name of that register is missleading, because it is not actually a
random value, but it is rather decremented on each instruction\cite{heiserAnatomyHighPerformanceMicrokernel}.\\
It is not clear whether this is a sensible replacement strategy,
but it can be used to ensure that the same TLB slot is not used for every \texttt{tlbwr} if the implementation
does not provide for any further replacement strategy.\\
Some TLB entries can be protecte from this ''random'' replacement by setting a value in the \texttt{C0\_WIRED}
register. The value in this register represents a lower bound, protecting all TLB entries that lie below it.\\
This is useful to keep some mappings in the TLB that are valid all the time.

% TODO should this be here? We should probably have a discussion about xv6 first
%Protected TLB entries in xv6 -> Trampoline page
Having a protected space of TLB entries can especially be useful for global mappings. xv6 employs such a global
mapping for every process and the kernel with the trampoline page.\todo{trampoline page should be explained in fundamentals/platform}

% Required Size of CSRs for TLB writing
\begin{figure*}[t]
    \centering
    \begin{bytefield}[bitwidth=\widefigurewidth/39,bitheight=\widthof{~PBMT~}, bitformatting={\tiny\bfseries}, boxformatting={\centering}]{39}
        \bitheader[endianness=big]{38,30,29,21,20,12,11,0} \\
        \bitbox{9}{VPN[2]} &
        \bitbox{9}{VPN[1]} &
        \bitbox{9}{VPN[0]} &
        \bitbox{12}{Page Offset}\\
    \end{bytefield}
    \caption[RISC-V Sv39 Virtual Address]{RISC-V Sv39 Virtual Address}
    \label{fig:theory:sv39va}
\end{figure*}

% -------------------------------------------------------------------------------------------------
% Foundation for RISCV extensibility
%   RISCV CSRs
%       CSRs instructions
%       Data to be communicated to the cpu
%       CSRs chosen Format
\subsection{RISC-V CSR} % -------------------------------------------------------------------------

RISC-V and its extensions currently provide no support for modifying the TLB in software.
RISC-V does however provide a lot of extensibility with the \texttt{Control and Status Registers} (CSRs).\\
CSRs are part of the RISC-V priviledged architecture and are provided by the \texttt{Zicsr} extension\cite{RISCVInstructionSet}.\\



% -------------------------------------------------------------------------------------------------

\subsection{TLB writing via CSRs} % ---------------------------------------------------------------

% Riscv can flush entries of a specific ASID only, this means that it keeps that information
% Somewhere in the TLB structure


% Qemu -> TLB structure, replacement

\todo{TLB CSR Format}

\todo{Hardware dependenc -> still dependent on the TLB structure of RISCV Sv39, VAs and PAs remain, as do PTEs}
% -------------------------------------------------------------------------------------------------
%                                            Exception Handler
% -------------------------------------------------------------------------------------------------
\section{Exception Handler}


% -------------------------------------------------------------------------------------------------
%                                            Mapping Function Design
% -------------------------------------------------------------------------------------------------
\section{Mapping function}
% Assumptions
%Design of Segmented VM softtlb -> Assumption: Process Memory allocation Model -> Only growing upwards
% Segmentation -> ASIDs -> HW support for Address Space Protection
% -> [jacob1998virtualissues]


% A look at input and output data
% Calculation of addresses

% xv6 Perspective and specifcs (but still theory!) -> Into Impl?
%   Special mappings -> special case in tlb manager (e.g. Trampoline)

% End of Theory: A simple mapping function - TLB Manager
%   TODO What does the reader need to know at this point, e.g. what does need to be decided 
%   in theory to create a TLB Miss Handler design??

\subsection{xv6 program memory model}
\subsection{Simple Segmented Mapping}


\begin{figure*}[ht!]
    \centering
    \includegraphics[]{figures/simple_mapping.pdf}

    \caption[Simple Mapping Scheme]{Simple Mapping Scheme}
    \label{fig:theory:simplemapping}
\end{figure*}
\chapter{Implementation}

\label{chap:impl}

% -------------------------------------------------------------------------------------------------
%                                           General TODOs
% -------------------------------------------------------------------------------------------------


\todo{Add references to documentation at appropiate places}
\todo{ add disclaimer, that most of this information was derived from reading the source code and may in some places not be accurate because the code is pretty complex and
    i have only been at it for a few months}

% TODO viruteller address raum eines xv6 processes
% Qemu implementation TLB vs Echter TLB
% -------------------------------------------------------------------------------------------------
%                                    Chapter structure overview
% -------------------------------------------------------------------------------------------------

%   Source overview
%     QEMU
%     xv6
%   Stage1 - Single Fixed address
%     Qemu - exception
%     xv6 - tlb triggerer
%   Stage2
%     Qemu - TLB CSRs
%     xv6 - Exception Handler, single address tlb fill
%       Exception vector entry -> state save, kernel stack
%   Stage3 - VM PTW with TLB miss handler
%     Qemu -> Extension to all addresses
%     xv6 -> tlb miss handler with ptw
%   Stage4 - Dynamic Segmentation allocation scheme (btw -> link to MIPS?)
%     Qemu -> No more changes needed
%     xv6 -> A lot of changes here
%   Debugging
%   Discussion on implementation
%       Concurrency
%       VM Features
% -------------------------------------------------------------------------------------------------

% High level description of the development steps and their reasoning

% -------------------------------------------------------------------------------------------------
% Section on the rationale for choosing this platform -> or just say that this was chosen and dont rationalize it

% -------------------------------------------------------------------------------------------------
% Introductory section
This chapter summarizes the implementation of the softtlb page-table-less virtual memory system.
The ordering of the sections in this chapter reflect the implementation process and will thus
be comprised of following steps:
\begin{enumerate}
    \item \textbf{TLB Miss Exception and Exception Triggerer} The first step is about
          the implementation of the TLB Miss Exception in the QEMU RISC-V emulator. On the xv6-side,
          a user-mode program is added to trigger a TLB Miss exception from the shell.
    \item \textbf{Exception Handling and TLB Writing} The second step implements the
          handling of the TLB Miss Exception, by implementing a machine mode exception handler.
          The RISC-V emulation is extended by two new CSRs that facilitate writing TLB from the
          software-side of things.
    \item \textbf{Software Page Table Walk for all Addresses} This step removes the restriction
          in the QEMU emulator to only throw exeptions for a specific address. In the exception handler,
          a page table walk is implemented to now create virtual to physical mappings for all addresses.
    \item \textbf{Segmented Memory Design using software-defined TLB Filling} In this step, the xv6 virtual memory system is completely
          replaced. The new design gets rid of the page table and only uses information present in
          registers to create virtual-to-physical mappings and to fill the tlb.\\
\end{enumerate}
Each section elaborates on both the xv6 side and the QEMU side of the implementation.\\
The final section describes debugging techniques that were used or can otherwise be useful for
similar implementations.

\begin{figure*}[t!]
    \centering
    \includegraphics[scale=1]{figures/sequence.pdf}
    \caption[Sequence of actions]{Sequence of action in case of a TLB Hit and Miss}
    \label{impl:sequence}
\end{figure*}


% First step in the development process
% -------------------------------------------------------------------------------------------------
%                                          SECTION STEP 1
% -------------------------------------------------------------------------------------------------
\section{TLB Miss Exception and Exception Triggerer }
There are two requisites the hardware needs to fulfill in order to do a software-managed TLB fill:
There needs a way to signal the operating system that a TLB miss occured and there needs to be some
sort of instruction that can be used to write TLB entries.\\
This step is about the former: Changing the QEMU RISC-V emulator to throw a newly defined \textit{
    TLB MISS Exception} whenever the TLB misses.\\
% Why only start with one fixed address?
Changing the whole system to start throwing a TLB Miss Exception on every virtual address would
make it very hard to debug both first tries at implementing a handler for the exception and the
exception throwing code in QEMU itself.\\
And not only would the exception be thrown as soon as virtual memory is activated in
\texttt{xv6-riscv:kernel/main.c}, the exception would be thrown
as soon as exceptions are activated and a memory access happens, because QEMU also uses the
\texttt{fill\_tlb} routine to fill the TLB with direct
virtual-to-physical mappings when no virtual memory is used. This speeds up the execution of
the dynamically-translated code, as it can directly
lookup addresses in the TLB using a fast path \cite{DeepDiveQEMU}.\\ \todo{add: dynamically-generated tcg code can
    directly use the TLB structures}
For this step of the implementation, the QEMU memory system emulation must be changed to throw a
TLB Miss Exception when a TLB lookup misses. To keep the system running as normal, this will
be done for only one hardcoded address, that is usually not used by xv6.\\
On the xv6 side, we need a user-level program, that accesses the specific address and thus prompts
the emulated hardware to throw a TLB Miss exception.


% -------------------------------------------------------------------------------------------------
\subsection{Address Selection}
The choice for an address to be used for testing the TLB Miss exception throwing is easy:
As shown in Figure \ref{impl:xv6layout} \cite{cox2011xv6}, the physical memory map of xv6 has a
area of ''Free memory'' that is managed by the physical memory allocator. The memory allocator
always gives out the next page starting from low to high addresses when a new page is requested
by kernel routines. The address \texttt{0x84fff000} was chosen as a testing address.\\
Note that the physical memory allocator in \texttt{kalloc.c} will actually touch an this (or
any address in the range from \texttt{0x80000000} to \texttt{0x88000000}), when initializing
the linked list used to keep track of the free pages \cite{cox2011xv6}.\todo{explain why this
    may be relevant or just leave it out? }
% does not need to be a accessible address now, but later we want to write and read from it

\begin{figure*}[t!]
    \centering
    \includegraphics[scale=.5]{figures/xv6_layout.pdf}
    \caption[xv6 memory layout]{The xv6 memory layout and how the kernel virtual address space is mapped on
        the physical address space. Taken from the xv6 book \cite{cox2011xv6}.}
    \label{impl:xv6layout}
\end{figure*}


\subsection{TLB Miss Exception in QEMU}
%TODO Explain what the riscv_cpu_tlb_fill function does originally - in detail
% TODO discussion and reference to READER/SPECIFICATION on what exception numbers and other constants can be used and are free

% Qemu RISCV  exception adding "tutorial"
%TODO Description of the TLB Exception implementation and how to generally add exceptions to the QEMU plattform
% TODO List of code locations where changes need to be added -> should be usable as back reference for implementing more exceptions

To properly test the implementation, the \texttt{tlb\_fill} function was replaced to throw the
TLB\_MISS exception for
one specified, page-aligned address and to continue on normally for every other address.
The implementation is outlined
in Listing \ref{lst:specialCaseTLBfill}.



% riscv_cpu_tlb_switch function 
%TODO is this listing really interesting?
\begin{lstlisting}[language=c,float=h!,
    caption={Alternative Implementation for the RISC-V tlb\_fill function with a special case to
    start testing TLB Miss Handler implementations.\\
    In line 11, a conditional branch is used to only trigger the exception when neither the
    Virtual Memory (as set in the \texttt{satp} \texttt{MODE} field) is bare nor the priviledge
    mode is the machine mode.\\
    If the virtual address is the hardcoded one, a TLB Miss exception is thrown, otherwise the
    original functions is called, which will perform a page table walk to find the mapping.},
    label={lst:specialCaseTLBfill}]
bool my_riscv_cpu_tlb_fill(CPUState *cs, vaddr address, int size,
    MMUAccessType access_type, int mmu_idx,
    bool probe, uintptr_t retaddr)
{
    RISCVCPU *cpu = RISCV_CPU(cs);
    CPURISCVState *env = &cpu->env;
    int mode = mmuidx_priv(mmu_idx);
    int vm = get_field(env->satp, SATP64_MODE);
    bool ret = false;

    if(!(vm == VM_1_10_MBARE || mode == PRV_M) &&
            address == (uint64_t)0x84fff000) {
        ret = riscv_cpu_tlb_miss_exception(cs,address,size,access_type, mmu_idx, probe, retaddr);
    } else {
        ret =  riscv_cpu_tlb_fill(cs,address,size,access_type, mmu_idx, probe, retaddr);
    }
    return ret;
}
\end{lstlisting}

% -------------------------------------------------------------------------------------------------


% This section describes the necesarry steps for adding the tlb miss exception to the qemu source
%in comparison with Page fault exceptions
To draw inspiration on how to implement a TLB Miss exception in QEMU, you can take a look at how
page fault exceptions are thrown.\\
Whenever a page fault exception is triggered, the TLB is checked first to see if there is a mapping
for the input virtual address \cite{QEMUSource2024}. Additionally, RISC-V cores provide the faulting
address of the page fault exception in the \texttt{mtval} register \cite{RISCVInstructionSet}.\\
The faulting address will also be necessary for handling the TLB Miss exception.


% -------------------------------------------------------------------------------------------------

% -------------------------------------------------------------------------------------------------

% Adding new Exception to Qemu RV target
\paragraph{Adding a new exception}to the QEMU emulator requires changes at a number of places. In the following,
the relevant code locations in the QEMU source \cite{QEMUSource2024} are shown.\\
This may be completely different for other targets, as the exception code is mostly target specific and
this implementation only looked at the RISC-V target.
\begin{itemize}
    \item \texttt{target/riscv/cpu\_bits.h} contains all CPU-definitions specific to the RISC-V target.
          There is also a enum called \texttt{RISCVException} which contains the number-codes for all RISC-V exceptions.
          In choosing a appropiate number for a new exception, one should consult the Privileged Architecture Specification \cite{RISCVInstructionSet}.
          There are specific exception code ranges that are designated for custom use. E.g. the codes 24--32 and 48--63.
    \item \texttt{target/riscv/cpu\_helper.c\:riscv\_cpu\_do\_interrupt\(\)} is the target-specific function
          for triggering interrupts. Here it suffices to add the new exception enum item to the switch case, when
          the new exception is similar in behavior to exising exceptions.\\
          Here the new exception is simply supposed to jump into an exception handler in the kernel. A lot of exceptions
          like page faults share that behavior.
    \item Finally, if the exception should be delegatable to supervisor mode or user mode, the n-th bit,
          with n being the exception code, needs to be set in the \texttt{DELEGABLE\_EXCPS} definition in \texttt{target/riscv/csr.c}.
          This enables the kernel to delegate the exception to another priviledge level by setting the appropiate
          bit in the \texttt{medeleg} and \texttt{sedeleg} CSRs.
\end{itemize}

The code shown in Listing \ref{lst:specialCaseTLBfill} will finally trigger the function shown in
Listing \ref{lst:exceptionThrow}.\\ After executing this function, QEMU will trigger a TLB Miss exception as soon
as it gets back to the main execution loop \cite{QEMUSource2024}.

\begin{lstlisting}[language=c,float=h!,
    caption={Setup-Code for raising a TLB Exception. The \texttt{cs-\textgreater exception\_index} variable needs
    to be set to the custom \texttt{TLB Exception} enum value. The \texttt{env-\textgreater badaddr} variable
    will end up in the \texttt{mtval} register. The address will be page-aligned first, by zeroing out the
    lowest 12 bits. This is used to encode the \texttt{mmu\_idx} into the faulting address. Why this is
    necessary is explained in Section \ref{sect:tlbwrite}},
    label={lst:exceptionThrow}]

    static void raise_tlb_exception(CPURISCVState *env, target_ulong address,
                                MMUAccessType access_type,
                                /*unnecessary?*/ bool pmp_violation,
                                bool first_stage, bool two_stage,
                                bool two_stage_indirect, uint8_t mmu_idx) {
        CPUState *cs = env_cpu(env);

        cs->exception_index = RISCV_EXCP_TLB_MISS;
        env->badaddr = ( address & ~( (1 << 12) - 1)) | mmu_idx;
        env->two_stage_lookup = two_stage;
        env->two_stage_indirect_lookup = two_stage_indirect;
    }

\end{lstlisting}

% -------------------------------------------------------------------------------------------------


\subsection{Exception Triggerer}
% TODO TLB exception trigger
To properly test the changes introduced to the QEMU emulator, there needs to be some way to
trigger a TLB exception.\\
By implementing this as a user-level program, the exception can be triggered using the xv6 shell.

\textbf{Adding a new user-level Program to xv6} only needs you to add a new \texttt{.c} file to the user subfolder
and to add the name of the generated binary ( name of \texttt{.c} file prefixed with a \texttt{\_})
to the list of user binaries in the makefile.\\
The new \texttt{.c} file only needs a \texttt{main} function and should also include the \texttt{user.h}
file to gain access to some preimplemented function and system call wrappers \cite{xv6source}.

The final exception triggerer may look something like this:

\begin{lstlisting}[language=c,float=h!,label{impl:excptTrigger},caption={\textbf{Exception Triggerer} Trying to
    load from a hardcoded address prompts the emulated hardware to trigger a TLB Miss Exception.}]
    #include "kernel/types.h"
    #include "user/user.h"

    void do_tlb_exc(void) {
        __asm__("li s2, 0x84fff000\n\t \
                lw s4, 0(s2)\n\t");
        register int *foo asm ("s4");
        printf("%x\n", foo);
        return;
    }
    
    int main(int argc, char *argv[]) {
        do_tlb_exc();
        //exit(0);
    }
\end{lstlisting}
The program first loads the hardcoded address into a register and then tries to load a word from this address.
If the implementation of the \texttt{TLB Miss Exception} was done correctly, the process will trap to the kernel
and the kernel will print out an error message, as it does for all exceptions that either have unknown exception
numbers or do not have a exception handler implemented \cite{cox2011xv6}.\\
If the exception was not properly implemented, the kernel would report a load page fault exception.

\todo{add xv6 shell output at this stage progression - should be a error printed by the kernel exception handler}
% -------------------------------------------------------------------------------------------------
%                                       END OF SECTION - STEP 1
% -------------------------------------------------------------------------------------------------



% Second step: Extending the tlb miss handling to be used for all addresses -> implementing a softvm ptw
% -------------------------------------------------------------------------------------------------
%                                         SECTION - STEP 2
% -------------------------------------------------------------------------------------------------
\section{Exception Handling and TLB Writing}
\label{sect:tlbwrite}

Now that the new \textit{TLB Miss Exception} can be triggered by a user-level program, there needs
to be an exception handler in the kernel that will create virtual-physical mappings and add them to the
TLB.\\
This section will first go into a general description to add new CSRs to the RISC-V QEMU emulation and
will then elaborate on the specific implementation for the TLB CSRs.\\
The section ends with the implementation of the exception handler.

% TODO Why does the mmuidx need to be encoded into the faulting address -> Explain


% -------------------------------------------------------------------------------------------------
\subsection{Adding CSRs to RISC-V/QEMU}

% Relevant Code locations
Following code locations are relevant for CSRs in the RISC-V/QEMU emulation source code \cite{QEMUSource2024}:
\begin{itemize}
    \item \texttt{disas/riscv.c} Contains a big switch case with all the CSR number to CSR name mappings.
          Name and number of new CSRs need to be added there.
    \item \texttt{target/riscv/cpu\_bits.h} contains definitions for all CSR numbers.
          While it is not strictly necessary to add another definition for new CSRs here, readability and maintainability
          of the code increases if a more descriptive definition name is used instead of a magic constant.
    \item \texttt{target/riscv/cpu\_cfg.h} contains a structure called \texttt{RISCVCPUConfig}. Every emulated RISC-V hart
          has this structure to expose all the extensions that the hart supports. \todo{hart already known?}
          The structure has a boolean flag for every extension that is currently supported by the emulator.\\
          New extensions should get their own flag in this struct.\\
          Similar entries also need to be added to the \texttt{isa\_edata\_arr} and \texttt{riscv\_cpu\_extensions} arrays in \texttt{target/riscv/cpu.c}.
    \item \texttt{target/riscv/csr.c} contains the implementation for all CSRs. The \texttt{riscv\_csr\_operations csr\_ops[]}
          array is essential for adding callback functions to CSR numbers.\\
          For every new CSR, a struct of the type \texttt{riscv\_csr\_operations} must be added to that array using the
          CSR number as an index. This struct is comprised of multiple function pointers, which deal with
          \begin{itemize}
              \item Checking if the hart implements the CSR
              \item Reading from the CSR
              \item Writing to the CSR
              \item Combined read/write
              \item 128 bit read/writes
          \end{itemize}
\end{itemize}
%Implementation of CSRs

As previously mentioned, the CSRs have some index ranges for new, custom CSRs. For the implementation
of TLB write CSRs, the indexes \texttt{0xBEE} and \texttt{BFF} have been selected.\\
Using these constants and the steps above, two new instructions\todo{are these actually instructions} can be realized.

\subsection{CSR Callback Implementation} Apart from the above mentioned steps to add new CSRs to the emulator, the
main logic of the implementation is in the callbacks referenced in \texttt{target/riscv/csr.c}.\\
The implementation of these callbacks is strongly dependent on the structure of the data that is written to
the CSRs. Fundamentally, these callbacks act as a bridge between the exposed ISA and the implementation of
that instruction set in software.\\
As previously mentioned, two new CSRs will be needed to implement the TLB-writing. \todo{ PUT THIS IN THE THEORY PART about CSR format and number: In theory, it would
    also be possible to implement the functionality using only one CSR, by just taking the faulting address from the mtval register, but
    this paper focuses on the proof of concept of the design and not on the optimization }.\\
The implementations
of the write-callbacks look as follows:


% Callbacks

\begin{lstlisting}[language=c,float=h!,
    label={lst:tlbh}]
    static RISCVException write_tlbh(CPURISCVState *env, int csrno, target_ulong new_val)
    {
        env->tlbh = new_val;
        return RISCV_EXCP_NONE;
    }
\end{lstlisting}
The implementation of the \texttt{tlbh} CSR write, does not do anything else but saving the value
that is written to it to the environment of the CPU.\\
This is because the theory specifies\todo{is it specified yet?}, that the TLB entry will only
be written to the TLB when the write to the \texttt{tlbl} CSR has succeeded.

\begin{lstlisting}[language=c,float=h!,
    label={lst:tlbl}]
    static RISCVException write_tlbl(CPURISCVState *env, int csrno, target_ulong pte)
    {
        
        target_ulong tlb_size = TARGET_PAGE_SIZE;
        
        CPUState *cpu = env_cpu(env);
        vaddr addr = env->tlbh;
        hwaddr paddr = ((pte & ~(PTE_RESERVED)) >> 10) << 12;

        int mmu_idx = addr & (tlb_size - 1);

        int prot = pte & (PTE_R | PTE_W | PTE_X | PTE_V );

        addr &= ~(tlb_size - 1);
        paddr &= ~(tlb_size - 1);

        tlb_set_page(cpu, addr, paddr, prot, mmu_idx, tlb_size, false);
        
        env->tlbh = 0;
        env->tlbl = 0;

        return RISCV_EXCP_NONE;
    }
\end{lstlisting}
\todo{all these code lines need a theoretical foundation in the fundamentals chapter (BTW, the format
    still adheres to PTEs!)}
The value written to the \texttt{tlbl} CSR adheres to the same format as the RISC-V Sv39 PTEs (As shown in section \ref{fund:sv39}).

To get the page-aligned physical address and to get rid of the access bits stored in the lower 10 bits,
the value will first be right-shifted by ten and then left shifted by 12 bits.\\
The top most bits are specified to be all zero, as explained in the fundamentals chapter \cite{RISCVInstructionSet}.

In line 10, the \texttt{mmu\_idx} is extracted from the lowest 11 bits of the virtual address. This is
necesarry, because QEMU uses up to 16 different MMU modes with dedicated TLBs \cite{QEMUSource2024}.
Whenever QEMU performs a TLB lookup, it does so in a specific MMU mode. This MMU mode is clear when
a TLB entry is retrieved, it is however not clear when a TLB entry is written.
To still fill the correct TLB, the \texttt{mmu\_idx} is transfered to the exception handler as part
of the faulting address in \texttt{mtval} and then back to the emulator via the TLB write CSRs.

The following lines deal with extracting the protection bits from the PTE and with page-aligning
the virtual and physical addresses. Finally, a preexisting QEMU function is invoked to add
a new entry to the emulated TLB and the CSR values are cleared.\\
The return value \texttt{RISCV\_EXCP\_NONE} indicates that nothing out of the ordinary happend.

This is all that needs to be done to add CSRs for TLB filling to the QEMU RISC-V emulator.


\todo{discussion of implementation at the end of this chapter: shortcommings, comparison QEMU - MIPS, comparison xv6 new vm - other vm -> eval??}

% -------------------------------------------------------------------------------------------------

\subsection{TLB Miss Exception Handler}
% Previous state of xv6 machine mode trap handler
%   -> Only for Timer Interrupt
%   -> Small amount of saved registers
% TODO TLB fill manager
% Changes -> Vectored mode to keep timer interrupt as is, just need to move it
With capabilities to write TLB entries in place, an effective exception handler can be implemented.

\textbf{xv6 Machine-Mode Trap Handler} The xv6 machine-mode trap handler only deals
with the timer interrupt. All other interrupts and exceptions are delegated to the supervisor
mode. This allows the trap handler to be very small and very specific to the timer interrupt \cite{cox2011xv6}.
It thus only needs to store two registers two memory to make enough room in the register file
to reset the timer and invoke the scheduler.\\
Adding another trap to be handled adds more complexity, as the trap number needs to be checked
and the code needs to branch to the right routines.

But it can also be completely avoided to touch the timer code at all. xv6 uses trap vectoring
mechanism in \textit{direct} mode.\\ In direct mode, all traps jump to the same address.
Using the \textit{vectored} mode makes all \textit{exceptions} jump to the address set
in the \texttt{mtvec BASE} field but all \textit{interrupts} are set the program counter
to \texttt{BASE} plus four times the interrupt cause \cite{RISCVInstructionSet}.

So to add machine mode exception handlers with disrupting the existing code as little as possible only
requires changing the trap vectoring mode to \texttt{vectored} mode and moving the timer interrupt code
to the correct offset.

\begin{lstlisting}[language=diff]
-    w_mtvec((uint64)timervec);
+    w_mtvec((uint64)mtvec_vector_table | 0x1);
\end{lstlisting}

The changes for changing the mode are trivial. Only the one bit indicating the \texttt{vectored} mode
needs to be set.\\
Placing the timer vector at the correct offset from the default trap handler vector can be achieved by
filling the space between the base address and the timer vector with no-operations. The same could be
achieved with linker directives.\\
The interrupt number for the timer interrupt is \texttt{0x7}, so thats puts the timer interrupt vector
at a positive \texttt{0x1c} offset from the base address.

\begin{lstlisting}[language={[RISC-V]Assembler},float=h!,
    label={lst:defaultTrapHandler}, caption={Vectored Trap Handler Routine}]
mtvec_vector_table:
IRQ_0:
        j default_exception_handler
        nop
IRQ_1:
        j default_vector_handler
        nop
IRQ_2:
        j default_vector_handler
        nop
IRQ_3:
        j default_vector_handler
        nop
IRQ_4:
        j default_vector_handler
        nop
IRQ_5:
        j default_vector_handler
        nop
IRQ_6:
        j default_vector_handler
        nop
IRQ_7:  # timer handler
        j timervec
\end{lstlisting}

% TODO verifying the placement of addresses and symbols using the name utility

RISC-V provides 4 bytes for every interrupt request (IRQ) and the default trap handler. This is not enough
to implement proper trap handling, so these 4 bytes are typically spent jumping to a more elaborate trap
handling routine.

Now any machine-mode exception (that is not delegated to a lower-priviledge mode) sets the program counter
to the address of the \texttt{mtvec\_vector\_table} label. The next jump brings the execution into
the default exception handler.
The timer interrupt will jump directly to the \texttt{IRQ\_7} label.

\textbf{Store/Load Sandwich} The idea behind the \textit{softtlb} design is to not use as many of the five
memory references modern architectures need for a page table walk \cite{intel5LevelPaging5Level2017}. But
for a first proof of concept, it is helpful to not have to think about which registers to use and to not optimize
away all unnecesarry memory references before the design even works.\\
That is why the first thing the default exception handler will do is to store the current state of the register
file.\\
This is essential so that, once returned to the context that triggered the TLB miss, execution can continue
as if nothing happened.\\
After storing the state and running the exception handler, the state must again be loaded into the registers.

\todo{use minipages to have text and listing side by side ? https://latex.org/forum/viewtopic.php?t=6652}
\begin{lstlisting}[language={[RISC-V]Assembler},float=h!,
    label={lst:defaultTrapHandler}, caption={Store/Load Sandwich}]
default_exception_handler:

    csrrw a0, mscratch, a0
    sd sp, 40(a0)

    ld sp, 48(a0)
    csrrw a0, mscratch, a0

    # make room to save registers.
    addi sp, sp, -256

    # save the registers to the hart stack
    sd ra, 0(sp)
    sd gp, 16(sp)
    ...
    sd t5, 232(sp)
    sd t6, 240(sp)

    addi s0, sp, 256 # Set Frame Pointer

    # ??
    # csrr t0, sscratch
    # sd t0, 8(sp)

    # TODO Pass args to fx

    # csrr a0, mtval
    # csrr a1, satp

    call machine_default_exception_handler 


    # restore registers from hart stack
    ld ra, 0(sp)
    ld gp, 16(sp)
    ...
    ld t5, 232(sp)
    ld t6, 240(sp)

    addi sp, sp, 256
    
    csrrw a0, mscratch, a0

    sd sp, 48(a0)
    ld sp, 40(a0)

    csrrw a0, mscratch, a0
    mret

\end{lstlisting}
\todo{side by side}
Most of the load/stores have been omitted to not flood the listing.\\
In between the store and load blocks, the \texttt{machine\_default\_exception\_handler} C function is called. This is
where the implementation differentiates between the different exception numbers and calls exception handlers
specific to the exceptions.\\
In xv6, all other exceptions but the new TLB Miss exception are handled in supervisor mode, so there are
no handlers for other exceptions.

\begin{lstlisting}[language={C},float=h!,
    label={lst:defaultTrapHandler}, caption={Exception Switch-Case Statement}]
    void machine_default_exception_handler() {
        switch (r_mcause()) {
            case NONE:
            case INST_ADDR_MIS:
            case INST_ACCESS_FAULT:
            case ILLEGAL_INST:
            case BREAKPOINT:
            case LOAD_ADDR_MIS:
            case LOAD_ACCESS_FAULT:
            case STORE_AMO_ADDR_MIS:
            case STORE_AMO_ACCESS_FAULT:
            case U_ECALL:
            case S_ECALL:
            case INST_PAGE_FAULT:
            case LOAD_PAGE_FAULT:
            case STORE_PAGE_FAULT:
            default:
                unhandled_exc(r_mcause(), r_mtval());
                break;
            case TLB_MISS:
                tlb_handle_miss(r_mtval(), r_satp());
                break;
        }
    }
\end{lstlisting}

Finally, the \texttt{tlb\_handle\_miss()} functions is called with the faulting address and the \texttt{satp} register
value as arguments.\\
The initial implementation simply calls the new \texttt{tlbh} and \texttt{tlbl} CSR instructions with and then
returns.
% Description of tlb handler and call of tlbh and tlbl
\begin{lstlisting}[language={C},float=h!,
    label={lst:defaultTrapHandler}, caption={Simple TLB Miss Exception handler for a single fixed address}]
void tlb_handle_miss(uint64 addr, uint64 usatp) {
  //uint64 *paddr = kalloc();
  w_tlbh(addr);
  w_tlbl((uint64)addr+(uint64)0x1000);
  return;
}
\end{lstlisting}

Note that this implementation does not yet use the format for the \texttt{tlbh} and \texttt{tlbl} CSRs specified
in the theory section \todo{reference to right theory section}.\\
Merely the faulting address and the intended mapping are written to the registers.

% HERE TODO explain the implication for implementation
\subsection{Testing}
Listing \ref{lst:defaultTrapHandler} already shows a part of the setup to test that the mapping actually worked.\\
When the mapping worked, it should be possible to read a value from the mapped page.\\
The user-level exception triggerer already tries to do just that.\\
With the mapping working, the output of the program looks as follows:

\todo{add console output}

The value \texttt{0xDEADBEEF} that was read from the mapped page is not a coincidence:\\
For testing purposes, the page \texttt{0x85000000} was filled with \texttt{0xDEADBEEF} bit patterns.

The address for the physical ''testing'' page was deliberately chosen to be \texttt{0x85000000} and not
\texttt{0x84fff000} (like all the hardcoded testing addresses) to make sure that there is no accidental physical
mapping happening.\\


\todo{add xv6 shell output at this stage progression - no more exception printing, xv6 should read deadbeef now}

% -------------------------------------------------------------------------------------------------
%                                       END OF SECTION - STEP 2
% -------------------------------------------------------------------------------------------------


% Third step: Dynamic Segmented memory with a simple function and no page table
% -------------------------------------------------------------------------------------------------
%                                       STEP 3 - New VM
% -------------------------------------------------------------------------------------------------
\section{Software Page Table Walk for all Addresses}
After implementation step 2, we now have a modified QEMU RISC-V emulator, which will throw the new
TLB Miss exception when the TLB misses.\\
We have also extended the xv6 operating system to catch said exception and to create a virtual-physical
mapping for the faulting address.\\
The mapping is created by writing new TLB entries using newly implemented TLB CSRs.

Currently, all of this only works for one specific ''testing'' address. This step now elaborates on
how the scheme is extended to be used for all addresses.\\
To not change too much at once, the page table walk that would typically be done in hardware is
implemented as part of the TLB Miss exception handler.

\subsection{TLB Miss Exception for all Addresses}
The changes introduced to QEMU are simple: Essentially, only the condition checking if the given address is
the ''testing'' address needs to be changed.\\
The final implementation looks as follows:

\begin{lstlisting}[language=c,float=h!,
    caption={},
    label={lst:updatedTLBFill}]
    bool my_riscv_cpu_tlb_fill(CPUState *cs, vaddr address, int size,
                        MMUAccessType access_type, int mmu_idx,
                        bool probe, uintptr_t retaddr)
    {
        RISCVCPU *cpu = RISCV_CPU(cs);
        CPURISCVState *env = &cpu->env;
        int mode = mmuidx_priv(mmu_idx);
        int vm = get_field(env->satp, SATP64_MODE);

        //No address translation in m-mode
        if(vm == VM_1_10_MBARE || mode == PRV_M) {
            int tlb_size = 4096; //TODO Get from PMP?
            hwaddr pa = address;
            int prot = PAGE_READ | PAGE_WRITE | PAGE_EXEC;
            tlb_set_page(cs, address & ~(tlb_size - 1), pa & ~(tlb_size - 1),
                        prot, mmu_idx, tlb_size, true);
            return true;
        }
        riscv_cpu_tlb_miss_exception(cs,address,size,access_type, mmu_idx, probe, retaddr);
        return true;
    }
\end{lstlisting}

Note that in comparison to the previous implementation (as shown in figure \ref{lst:specialCaseTLBfill}), the
call to the original \texttt{riscv\_cpu\_tlb\_fill} function is not called anymore.\\
Either a TLB Miss exception is raised, or the TLB is filled with an identity mapping if either the current
priviledge mode is the machine mode or the virtual memory system is configured to use identity mappings only
\cite{RISCVInstructionSet}.

% -------------------------------------------------------------------------------------------------

\subsection{Software Page Table Walk}

Now that the TLB Miss exception is thrown for every faulting address, it also needs to be handled
for every address.\\
Even though the goal is to explore alternatives to virtual memory using page tables, we will
use a software implementation of the hardware page table walk. This is to verify that the
general setup including exception invokation, exception handling and TLB writing works as
expected.
Listing \ref{lst:softptw} shows the implementation.

\todo{The listing seems to be way broader than it needs to be}
\begin{lstlisting}[language=c,float=t,
    caption={\textbf{TLB Miss Exception Handler with Page Table Walk} 
    The \texttt{walk\_pt()} function walks the Sv39 page table with the base address
    encoded in the \texttt{satp} register. If a PTE with the valid bit set is found, the function
    returns the address encoded in the PTE.\\
    Otherwise, the function returns \texttt{0}.},
    label={lst:softptw}]
#define SATP2PA(satp) ((satp << 12) & ~(0xffull << 44))

pte_t *
walk_pt(uint64 satp, uint64 va)
{
  uint64 *pt = (uint64*)SATP2PA(satp);
  if(va >= MAXVA)
    panic("walk");
  for(int level = 2; level > 0; level--) {
    pte_t *pte = &pt[PX(level, va)];
    if(*pte & PTE_V) {
      pt = (uint64*)PTE2PA(*pte);
    } else {
      return 0;
    }
  }
  return &pt[PX(0, va)];
}

void tlb_handle_miss(uint64 addr, uint64 satp) {
    w_tp(r_mhartid());
    pte_t *pte = walk_pt(satp, addr);
    w_tlbh(addr);
    uint16 prot =  PTE_R | PTE_W | PTE_U | PTE_V;
    w_tlbl(*pte | prot);
    return;
}
\end{lstlisting}

\todo{To make it easier to understand -> Architectural overview of the memory path after each implementation step?}

If the \texttt{walk\_pt()} function returns 0, the virtual address lead to no valid PTE. This is a
page fault and would raise a page fault fitting the faulting instructions type of access \cite{tanenbaumOS}.
As xv6 does not handle page faults other than by killing the program \cite{cox2011xv6} this
paper will not go into more detail of raising and triggering page faults for the \texttt{softtlb}
design.


% -------------------------------------------------------------------------------------------------
%                                               STEP 4
% -------------------------------------------------------------------------------------------------

\section{Segmented Memory Design using software-defined TLB Filling}

After step 3, xv6's virtual memory system runs completely without using the (emulated) page table
walking state machine:
Instead of invoking the page table walker, a TLB miss triggers a exception. This exception is handled by the
operating system kernel and a TLB entry for the faulting address is generated.\\
Yet at this point, we have not really won anything. We only moved the page table walking to the software side,
which would certainly be slower at traversing the page table than a real MMU \cite{jacob1998look}. And that
even if we ignore the expensive context switch that needs to be performed when the exception handler is invoked.\\

While we definitely did not gain any performance, there is now the opportunity to modify the virtual memory system
completely in software. We do not need to adhere to the rigid structures \cite{tanenbaumOS} of the memory management hardware anymore.
We gained a great deal of flexibility.

The rest of this section explores a first idea of how a table-based virtual memory system can be
replaced by essentially a simple function that does not require expensive memory accesses.

The initial idea, as presented in the theory chapter, is a segmented memory design. It splits the total
available memory into parts of equal size, also called segments.\\
Each new process acquires one of those segments, if there are any left. If none are left, process creation fails.

\todo{basic assumption to make this work: xv6 process memory model -> should be mentioned here and explained in detail in theory}
% TODO Concept of AS used here -> distinct from hard ware mechanism, but could be used for it
\subsection{Address Spaces}
As with segmentation, the available physical memory is split into evenly sized portions\todo{is it? Cite!},
from now called address space. When a new process is started it tries to acquire an address space by
checking whether there is a free one and then claiming that free address space.
The number of address spaces will thus limit the number of concurrently running programs.

To keep track of which process has which address space, the kernel keeps an integer array of process IDs.
This array has a size equal to the number of address spaces (NAS).\\
For a newly created process to acquire an address space, the kernel will (synchronized by a spinlock \cite{cox2011xv6})
iterate through the array to find a array slot with the value 0. It then writes the process ID into that array.\\
The process creation fails if there are no more array slots set to 0.
THE address space ID (ASID) acquired by the process is also written to the processes control block (PCB).

NAS is set via a preprocessor definition and thus set at compilation time. More elaborate implementations
could also allow dynamic resizing of address spaces to provide space for more processors.\todo{configuration
    towards ones needs -> embedded systems may need a fixed number of processes, or have a maximum amount}

Using the ASID, the address range for a process can be calculated very easily. The macros in listing \ref{TODO LISTING}
are used in different parts of the code to calculate different addresses related to an Address Space.

\todo{Add code listing (its in the tex comments)}
% TODO Macros for AS calculation
% #define MAX_AS_MEM ((((PHYSTOP - (PGROUNDUP((uint64)end))) / (NAS + 1)) >> 12 )<< 12)
% #define AS_START(asid) (PGROUNDUP((uint64)end) + ((asid) * MAX_AS_MEM))
% #define AS_N(index, asid) (PROC_START(asid) + (index * (0x1000)))
% #define AS_END(asid) ((PGROUNDUP((uint64)end) + ((asid + 1) * MAX_AS_MEM))-0x1000)

\subsection{Mapping function for Segmented Memory}
The ASID of a process is the foundation for calculating the virtual to physical mapping in the exception handler.
The only other information required is the faulting address.
\todo{theory needs to explain that the asid is transfered via SATP reg}
\\
The ASID can be used to calculate the actual physical address at which the Address Space starts.
The faulting address, which would, according to the xv6 process memory model, be a address in the low
end of the virtual memory space, will provide the offset into the Address Space.

Listing \ref{TODO LISTING} shows the implementation for the TLB Miss exception handler.\\

\todo{Add code listing (its in the tex comments)}
\todo{caption explaining the code}
% TODO new and improved exception handler
% #include "defs.h"
% #include "param.h"
% #include "types.h"
% #include "riscv.h"
% #include "memlayout.h"

% #define SATP2ASID(satp) ((satp << 4) >> 48)

% extern char trampoline[];
% paddr get_mapping(vaddr addr, uint16 asid) {

%   //special case for trampoline, which is in every address space at the same address
%   // (except for the physical one!)
%   if(PGROUNDDOWN(addr) == TRAMPOLINE) {
%     return (uint64)trampoline;
%   }
%   if(asid == 0) {
%     //Kernel -> direkt mapping
%     //TODO special mappings
%     return addr;
%   } else {
%     //mappings for processes
%     //TODO trapframe and trampoline mappings
%     if(PGROUNDDOWN(addr) == TRAPFRAME) {
%       return TRAPFRAME_FROM_ASID(asid);
%     }

%     //Base case 
%     uint64 vpage = PGROUNDDOWN(addr);
%     if(vpage >= MAX_AS_MEM - (0x1000 * 4)) {
%       //Error, address out of range
%       panic("tlb_manager: address to high!");
%     }
%     return AS_START(asid) + vpage;
%   }
%   return 0;
% }

% void tlb_handle_miss(vaddr addr, uint64 satp) {
%   w_tp(r_mhartid()); //fix problems with locks based on cpuid()

%   //TODO Locks!
%   //TODO buffering printer, only prints completed lines
%   //uint16 mmuid = addr & 0xfff;
%   vaddr addr_no_mmuid = addr & ~0xfff;
%   //printf("tlb_manager: addr=%p satp=%p mmuid=%d\n", addr_no_mmuid, satp, mmuid);
%   paddr pa = get_mapping(addr_no_mmuid, SATP2ASID(satp));

%   w_tlbh(addr);
%   //TODO all access rights for now
%   uint16 prot =  PTE_R | PTE_W | PTE_X | PTE_V | ((SATP2ASID(satp) != 0) ? PTE_U : 0);
%   w_tlbl(PA2PTE(pa) | prot);
%   return;
% }


% TODO it is not enough to simply change the tlb miss manager, because other
%   parts of te OS still use the VM system for stuff
%       What are these parts that interact with the os?
%       How can be made it fit this scheme?

\subsection{Special Mappings}
\todo{has the new mapping and trampoline in general been explained in theory?}
xv6 needs a trampline page mapped into the address space of the process for system calls to work properly (see Chapter \todo{ref theory}).
In both the kernel and in the user processes, it is mapped to the highest possible, virtual page.\\
If a address in this page is encountered, the handler will map it to the physical address of the trampoline page.
The handler takes the same approach for the \texttt{TRAPFRAME} page set just below the trampoline.

\subsection{Further Changes to the OS}
Other than the exception handler, other parts of the OS need to be changed to fit the new memory managment scheme.


% TODO dont list the code that needs to change but rather the logical parts, and how they logically need to change
% Rather from   concept -> code  than the other way around
The following parts of the code interact directly with the virtual memory system:
\begin{itemize}
    \item \texttt{kernel/exec.c}
    \item \texttt{kernel/main.c} calls the \texttt{kvmmake()} function to initialize a page table for the kernel.
    \item \texttt{kernel/proc.c}
    \item
\end{itemize}

%TODO Change to being about the kernel memory, and how it uses the physical memory allocator
\textbf{Physical Memory Allocator} The virtual memory subsystem relies on the physical memory
allocator in \texttt{kalloc.c} to get new pages \cite{cox2011xv6}. It is, however, not the only
part of the system that needs memory on demand. Some modules like the file system and de disc driver
need memory as well. To still service these parts of the code, the memory managed by the physical
allocator is simply reduced to leave space for the memory segments.

The new allocation scheme statically assigns a fixed portion of the available physical memory
to each process. The size is determined by the size of physical memory and the maximum number
of processes.
kalloc() will then use the process id, the current size of the process and the requested
increase in size to determine the next page for the process.
\todo{assumptions on the process model -> allocation works only for processes that grow linearily and do not use pages in between or at the top}

\textbf{changes to systemcalls? Esp. Fork, exec, sbrk}

\todo{process creation -> Assigning ASID etc}


% Trampoline, Trapframe

% -------------------------------------------------------------------------------------------------

% \subsection{Special cases}
% \paragraph*{Kernel direct mapping}
% \paragraph*{MMIO} -> Kernel
% \paragraph*{Trampoline}


% -------------------------------------------------------------------------------------------------
%                                    END OF SECTION - Step 4
% -------------------------------------------------------------------------------------------------



% TODO Should this be an appendix?
% -------------------------------------------------------------------------------------------------
%                                         SECTION - Debugging
% -------------------------------------------------------------------------------------------------

\section{Debugging}
Wenn man gleichzeitig Änderung am Qemu Emulator und an xv6 durchführt, kann es schwierig sein
herauszufinden wo der Fehler liegt. Hier wird zunächst einmal beschrieben wie man Qemu und
xv6 einzeln debuggen kann und darauf wird ein gemeinsamer Debugging Prozess aufgezeigt.
Das kann zum Beispiel nützlich sein wenn man versucht den Code Pfad zwischen Software und
Emulierter Hardware nachzuverfolgen.\\
Hier wird keine Anleitung zum benutzen von GDB geliefert, sondern nur ein paar eigenheiten
und tricks die beim debuggen des Codes für diese Arbeit nützlich waren. Vielleicht sind
diese ja auch in anderen Szenarios nützlich.

% -------------------------------------------------------------------------------------------------

\subsection{xv6}
Das Makefile im xv6-riscv source repository verfügt über die Regel qemu-gdb um xv6 in qemu
mit einem GDB server zu starten. In einem anderen Terminal kann man dann gdb mit der binary
die man debuggen möchte starten und dann mit \texttt{target remote:<port>} eine verbindung
zum gdbstub herstellen.\\
Ein Kommentar im makefile legt zwar nahe, dass die Makefile rule zum Debuggen vom user mode
gedacht ist, aber tatsächlich hält einen nichts auf auch den kernel zu debuggen. Das ist
für Änderungen am Memory system auch sehr nützlich.\\
\textbf{Debugging user mode} Für das debuggen eines user mode programms will man oft bei
der main Funktion anfangen. Breakt man auf die Funktion wird ein Breakpoint auf eine eine recht
niedrige virtuelle Addresse gesetzt. Hat man den Breakpoint schon gesetzt bevor man überhaupt
den Kernel gestartet hat kann es dann beim start sein, dass man direkt wieder stoppt.
Das liegt dann daran, dass man sich gerade im Qemu boot ROM befindet, der von Qemu in
das emulierte Memory Layout in den Bereich von \texttt{0x0 - 0x1000} gemappt wurde.
Der debugger unterscheidet hier nicht zwischen virtuellen und physischen Addressen und
schaut lediglich was im PC steht.
\todo{das satp PPN field könnte noch für weitere infos benutzt werden (Konfiguration der vm für den process)}

\textbf{Debugging the kernel} Nach dem ersetzen des virtuellen Memorysystems hat sich beim
debuggen des Kernels ein interessanter effekt bemerktbar gemacht: Sobald in der
\texttt{kvminithart()} funktion das \texttt{satp} register gesetzt wird kann GDB den code zu der
aktuellen addresse (und allen folgenden) anzeigen.\\
Das ist dem Wert des \texttt{satp} registers geschuldet. Nach dem Umbau des VM systems enthält
dieses nicht mehr die \texttt{PPN}. Nur noch das \texttt{MODE} und \texttt{ASID} Feld sind
für das neue system von bedeutung. Es gibt auch gar keine Page table mehr auf die die PPN zeigen
könnte.\\
Dieses Verhalten lässte vermuten, dass der von qemu implementierte GDBstub den hardware-emulierten
page walk macht um an die physischen addressen und damit an den Code zu kommen. Das funktioniert
natürlich nicht ohne page table.\\
Eine Möglichkeit den kernel wieder ordentlich in gdb debuggen zu können, wäre es die kernel
page table wieder so wie zuvor aufzubauen. Da aber die mappings im Kernel alle direkte mappings
sind, reicht es auch einen PTE für die addresse \texttt{0x80000000} bei der der kernel code
anfängt, hinter der PPN in \texttt{satp} abzulegen.\\
% wie muss der PTE aussehen?
Im Code wurde dafür eine ganze Page wie folgt allokiert:
\begin{lstlisting}[language=c,float=h!,
    label={lst:fake_pt}]
    uint64 *pt = (uint64*)kalloc();
    for(uint64 i = 0 ; i < 512; i++) {
      pt[i] = 0xffffffffffffffff;
    }
    pt[2] = ((0x80000000 >> 2)| PTE_V | PTE_X | PTE_W | PTE_R);
\end{lstlisting}
Es würde allerdings auch reichen nur den Platz für einen einzelnen PTE
zu allokieren. Die PPN muss allerdings dann so gesetzt werden,
dass mindestens der Eintrag bei Index 2 vorhanden ist. Interpretiert man
die Adresse \texttt{0x80000000} als virtuelle Addresse, so ergibt sich
nämlich für das \texttt{VPN[2]} Feld der Wert 2.\\
Setzt man diesen Top-level page table entry als valide, so wird der Speicher
an dieser Stelle 1 GB Superpage behandelt. Das reicht aus um den Kernelcode
abzudecken und der Code kann beim Debuggen wieder wie gehabt gesehen werden.

\textbf{Verifying Addresses in Binaries} Implementing the vectored trap vectoring mode
required moving the interrupt handler code for the timer interrupt to a specific offset
from the address specified in the \texttt{mtvec} \texttt{BASE} field.\\
To verify the correct placment of the label, the unix \texttt{nm} tool is very useful:
It lists the symbols of a object file and their addresses.\\
Looking back at the code in listing \ref{lst:defaultTrapHandler}, the label \texttt{IRQ\_7}
is supposed to come exactly \texttt{0x1c} bytes after the \texttt{mtvec\_vector\_table} label.\\

The following call confirms the correct placement of the symbols:

% \begin{minted}{asm}
%     addi    a0, zero, 2
%     addi    a1, a0, 1
%     sw      a1, 0(a5)
%     addi    a1, zero, 42
%         \end{minted}

\todo{how to cite this}

\begin{lstlisting}[language={sh}]
    $ nm kernel/kernel | grep 'mtvec_vector_table\|IRQ_7'
    00000000800092cc t IRQ_7
    00000000800092b0 T mtvec_vector_table
\end{lstlisting}

\todo{forktest for testing number of address spaces}
% -------------------------------------------------------------------------------------------------

\subsection{QEMU Monitor}
Mit dem Qemu monitor lassen sich allerlei informationen über die laufende
emulation anzeigen lassen. Besonders interessant für diese Arbeit wäre der
monitor für die aktuell im emulierten TLB enthaltene Mappings. Gerade für
RISC-V war dieser allerdings nicht verfügbar.
\todo{Beschreibung der TLB strukturen früher im kapitel}

% -------------------------------------------------------------------------------------------------

\subsection{QEMU Record/replay}
QEMUs Redord/replay feature allows a deterministic replay of a machines execution by recording
all non-deterministic events that happend. This can be especially helpful to investigate bugs
that occur as a result of an interrupt or require special preconditions or timings to occur.\\
With the Record/replay feature, it is easy to recreate the situation in which a bug or unexpected
behavior occured.\\
For this project, especially the mouse and keybord input and the hardware clock are interesting
non-deterministic events that influence the execution of the system \cite[RecordReplayQEMU].


% -------------------------------------------------------------------------------------------------
%                                    END OF SECTION - Debugging
% -------------------------------------------------------------------------------------------------


% -------------------------------------------------------------------------------------------------
% -------------------------------------------------------------------------------------------------
%                                           END OF CHAPTER
% -------------------------------------------------------------------------------------------------
% -------------------------------------------------------------------------------------------------



% -------------------------------------------------------------------------------------------------
%                                   STUFF THAT NEEDS TO BE SORTED IN
% -------------------------------------------------------------------------------------------------




.% -------------------------------------------------------------------------------------------------


\begin{figure*}[ht!]
    \centering
    \includegraphics[scale=1.5]{figures/theory_sw_ptw.pdf}
    \caption[Software Page Table Walker]{Instead of transfering control over to the MMU to fill in the missing mapping,
        the TLB Miss exception invokes a software page table walker}
    \label{fig:theory:sw_ptw}
\end{figure*}





\begin{figure*}[t]
    \centering
    \begin{bytefield}[bitwidth=\widefigurewidth/56,bitheight=\widthof{~PBMT~}, bitformatting={\tiny\bfseries}, boxformatting={\centering}]{56}
        \bitheader[endianness=big]{55,30,29,21,20,12,11,0} \\
        \bitbox{26}{PPN[2]} &
        \bitbox{9}{PPN[1]} &
        \bitbox{9}{PPN[0]} &
        \bitbox{12}{Page Offset}\\
    \end{bytefield}
    \caption[RISC-V Sv39 Physical Address]{RISC-V Sv39 Physical Address}
    \label{fig:theory:sv39pa}
\end{figure*}


\begin{figure*}[t]
    \centering
    \begin{bytefield}[bitwidth=\widefigurewidth/64,bitheight=\widthof{~PBMT~}, bitformatting={\tiny\bfseries}, boxformatting={\centering}]{64}
        \bitheader[endianness=big]{63,60,59,44,43,0} \\
        \bitbox{4}{Mode} &
        \bitbox{16}{ASID} &
        \bitbox{44}{PPN} \\
    \end{bytefield}
    \caption[RISC-V Sv39 \texttt{satp} CSR]{RISC-V Sv39 \texttt{satp} CSR}
    \label{fig:theory:sv39satp}
\end{figure*}



% Source Description Stuff - maybe extra chapter, or in way less detail


% -------------------------------------------------------------------------------------------------

\todo{merge this with overview and provide this sort of explanation as a introduction to the implementation part}
\section{Programming Platform}
\subsection{QEMU}
The \texttt{softtlb} design approach requires the CPU to throw an exception when the TLB misses. There is
currently neither a RISC-V platform that supports nor a extension to the RISC-V that specifies that behavior.\\
The logical consequence is to use an emulator and implement the required functionality.\\
This paper uses the QEMU emulator as a foundation for the implementation. QEMU supports a big range of
different platforms and is thus not the simplest source to modify. Additionally, detailed documentation
on internal structures is sparse.\\
However, QEMU does support a lot of features and extensions of the RISC-V target and is quite performant
\cite{bellard2005QEMU}.

% -------------------------------------------------------------------------------------------------

%This section is important for the rationale on choosing a format of the CSRs
%Vielleicht sollte dieser Teil einfach erst im hinblick auf replacement policies Dikutiert werden
% In meinem Programmiermodell ist ja der TLB nichts weiter als ein Indiziertes array von daten, 
% Weitere Aspekte sind ja hier erst mal out of scope
\subsubsection{TLB}
% What type of cache modeled? - direct mapped, associative?

\todo{description of QEMU TLB structure in fundamentals, theory or implementation?}
% Does the QEMU tlb implementation impact the theory? -> It should not
%   It only has impact once we have to decide on a specific format for the tlbh and tlbl csrs
%   But I can put all the theoretical parts into theory, so ASID etc
%   and the modification of the theory based on concrete hardware can come into the impl chapter
%       -> mmuidx and stuff


% -------------------------------------------------------------------------------------------------


\subsection{xv6-riscv}
xv6 is a simple teaching operating system loosely following the design of Unix Version 6 \cite{cox2011xv6}.
It is used to teach an operating systems course and does not contain the complexities of real-life
operating systems.

% -------------------------------------------------------------------------------------------------
%                                    SECTION Source Description
% -------------------------------------------------------------------------------------------------
\section{Source Overview}
The implementation consists of two fundamental parts:
The extension of RISC-V ISA emulated by QEMU and the implementation of a
TLB Miss Handler.

% -------------------------------------------------------------------------------------------------

\subsection{QEMU}
\texttt{cputlb.c} found in \texttt{accel/tcg} contains the TLB-handling logic for all
emulation targets; from there, target-specific functions for TLB management are called.
The target-specific functions are implemented in \texttt{target/riscv}.\\
Every target needs to implement the functions in the \texttt{TCGCPUOps} struct in
\texttt{include/hw/core/tcg-cpu-ops.h}. This struct is part of the glue that connects the target-independent
Tiny Code Generator (TCG) backend with the target-specific functions and structures.

% -------------------------------------------------------------------------------------------------

\todo{sequence diagram for function calls accross the QEMU repository}

% -------------------------------------------------------------------------------------------------

% TCGCPUOps struct
% TODO Ausgangspunkt sollte die mmu_lookup1 funktion sein!
The \texttt{TCGCPUOps} struct includes a function pointer named \texttt{tlb\_fill}. The function is called
when MMU lookups to the TLB and QEMU's victim TLB fail.
For the riscv target, the function \texttt{riscv\_cpu\_tlb\_fill} is assigned to the \texttt{tlb\_fill}
function pointer in \texttt{target/riscv/tcg/tcg-cpu.c}.\\
The implementation for \texttt{riscv\_cpu\_tlb\_fill} is in \texttt{target/riscv/cpu\_helper.c}. This is a common
file to be found in a target-specifc directory and contains general functionality to implement the \texttt{TCGCPUOps} struct.\\


\todo{explain PTE}

% -------------------------------------------------------------------------------------------------

\subsection{xv6}



% -------------------------------------------------------------------------------------------------
%                                END OF SECTION - Source Description
% -------------------------------------------------------------------------------------------------


\chapter{Evaluation}

\label{chap:eval}

\section{performance?}
\subsection{Problems with measuring performance of an emulated system}
\chapter{Future Work}
\label{chap:fut}

% -> Decreasing the footprint of the interrupt handler to inline it
% Source: In-Line Interrupt Handling for Software-Managed TLBs

\chapter{Conclusion}

\label{chap:conclusion}

\cite{choudhuri2005software} % Potential use case?


% CONCLUSION: IT WORKS
This paper presents the theory behind and the modifications of the xv6 operating system running on the QEMU/RISC-V platform to transform the fundamental control flow of the virtual memory path. The result is a simple platform is a foundation to start experimenting with mapping functions that replace the typical page table structure used by commodity hardware.

The paper also presents a mapping function implemented on that platform, realizing a ( quite restricted ) virtual memory system, that works completely without orthodox paging bookkeeping structures. The changes are completely transparent  to the system call interface of the operating system and do thus not require changes to programs running on the xv6 operating system.



%----------------------------------------------------------------------------------------
%	THESIS CONTENT - APPENDICES
%----------------------------------------------------------------------------------------

% By using input instead of include for the chapters we are able to move the following line here
% Therefore the addition before the last chapter is not necessary anymore.

% Call the following chapters "Appendix" inside the table of contents
\addtocontents{toc}{\string\def\string\chaptername{Appendix}}

\appendix % Cue to tell LaTeX that the following "chapters" are Appendices

% Ensure proper section numbering in appendix, e.g., A.1, A.2, B.1, …
\renewcommand{\thesection}{\thechapter.\arabic{section}}
\renewcommand{\thesubsection}{\thesection.\arabic{subsection}}
\renewcommand{\thesubsubsection}{\thesubsection.\arabic{subsubsection}}

%%% CHANGES NEEDED HERE
%
% Include the appendices of the thesis as separate files from the Appendices folder
% Uncomment the lines as you write the Appendices

% Appendix A

\chapter{Appendix about something}
\label{appendixa}


\chapter{More stuff} \todo{change chapoter name}
\label{appendixb}

%\include{appendices/appendixC}



%----------------------------------------------------------------------------------------
%	BIBLIOGRAPHY
%----------------------------------------------------------------------------------------

% Bibliography has no wide margins:
\newgeometry{
  inner=2cm, % Inner margin
  outer=2cm, % Outer margin
  marginparwidth=0cm,
  marginparsep=0mm,
  bindingoffset=.5cm, % Binding offset
  top=1.5cm, % Top margin
  bottom=2.5cm, % Bottom margin,
  includehead,
  includefoot
  % showframe, % Uncomment to show how the type block is set on the page
}

\addchap{References}

% enables two-column layout for bibliography
\setlength\columnsep{2em}
\begin{multicols}{2}
  \begin{refcontext}[sorting=nyt] % sort bibliography by last name, year, title
    \renewcommand*{\bibfont}{\small\RaggedRight}
    \linespread{1.0}\selectfont % increase linespread if desirBIB
    \printbibliography[heading=none]
  \end{refcontext}
\end{multicols}

%----------------------------------------------------------------------------------------

%----------------------------------------------------------------------------------------
%	DECLARATION PAGE
%----------------------------------------------------------------------------------------
\pagenumbering{gobble}
\begin{declaration}
  \addchaptertocentry{\authorshipname} % Add the declaration to the table of contents

  % TODO Change the declaration according as needed. *

  %\selectlanguage{ngerman}
  Ich erkläre hiermit gemä\ss\ \S~9 Abs.\,12 APO, dass ich die vorstehende {\thesistype}arbeit selbständig verfasst und keine anderen als die angegebenen Quellen und Hilfsmittel benutzt habe. Des Weiteren erkläre ich, dass die digitale Fassung der gedruckten Ausfertigung der {\thesistype}arbeit ausnahmslos in Inhalt und Wortlaut entspricht und zur Kenntnis genommen wurde, dass diese digitale Fassung einer durch Software unterstützten, anonymisierten Prüfung auf Plagiate unterzogen werden kann.

  \bigskip
  \bigskip

  \begin{tabular}{@{}l@{}}
    Bamberg, den \rule[-0.8em]{10em}{0.5pt} \\[2ex]
    ~
  \end{tabular}
  \hspace{\fill}%
  \begin{tabular}{@{}c@{}}
    \rule[-0.8em]{20em}{0.5pt} \\[2ex]
    \authorname
  \end{tabular}\hspace{\fill}




\end{declaration}

\end{document}
