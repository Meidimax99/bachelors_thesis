\chapter{Evaluation}

% -------------------------------------------------------------------------------------------------
%                                         Copy Paste from Discussion
% -------------------------------------------------------------------------------------------------

% Possible performance improvements:
% In-Line Interrupt Handling for Software-Managed TLBs -> reducing size
% of miss handler -> inlining it into reorder buffer

\section{Restrictions of the current design}
A segmented design has some restrictions compared to contemporary virtual memory
systems. In the following, important requirements to typical virtual memory
systems are presented and how the presented design can hold up.
\todo{How it fullfills these requirements now, how it may or may not fullfill currently
    unsupported requirements.}


%Flexibility limitation -> flexibility of the virtual memory system
% is still restricted by the tlb structure
% -> other approach jacobSoftwaremanagedAddressTranslation1997

% Vergleich MIPS Page table fast path

% Inside L4 Mips fastpath vs slowpath

% Harware erweiterung wenn die funktion parametrisierbar ist

% -------------------------------------------------------------------------------------------------
%                                          Discussion CP End
% -------------------------------------------------------------------------------------------------
\label{chap:eval}

% Qualitative Vergleiche

% Ohne caches -> Emualtion vs Real Hardware -> Warum keine quantitaive analyse

% Passt das in eine Cache line -> code localität

\section{cost analysis}
% Theoretische Kostenanalyse -> context switch in sw (ausblick inline?)
\section{performance?}
\subsection{Problems with measuring performance of an emulated system}
% Liedtke gpts?

\section{Memory System Requirements}
\cite{jacobSoftwaremanagedAddressTranslation1997}

% TODO discussion (extra chapter?) about whether fork makes sense with 
% this segmented allokation approach

% Instruction Count, Software Page Walk vs TLB manager
