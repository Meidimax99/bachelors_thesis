\chapter{Evaluation}
\label{chap:eval}


\section{Qualitative analysis}
\subsection{Cost}
% Context Switch
% -> Inline Handler + Vergleich mit größe de Heiser fast TLB miss handlers
% Annahme: Mein TLB miss handler/mapping function könnte sehr klein sein wenn sie 
% nur sehr gut optimiert wird und dann könnte man sie eventuell inlinen!
% Hardware Freeze on HW MMU Walk (but independent instruktions can go on -> issues of impl)

% Passt das in eine Cache line -> code localität


% Theoretische Kostenanalyse -> context switch in sw (ausblick inline?)
% -> Optimization -> Reserve Registers for kernel handler, but this wont be good 
% -> l4 20 year paper -> registeres important for optiization

% \section{Other operating system features in light of the segmented design}
% TODO discussion (extra chapter?) about whether fork makes sense with 
% this segmented allokation approach

% Instruction Count, Software Page Walk vs TLB manager

% switching to kernel evicts other cache entries for kernel in real hardware

% -------------------------------------------------------------------------------------------------
\subsection{Design Restrictions} % Generall Segmented Design shortcommings
% Allgemeine Discussion eines Segmentierten Designs
% \cite{skarlatos2020elastic}
% \cite{tanenbaumOS}
% \cite{denning1997before}??
% TODO [denningVirtualMemoryDenning1996] -> Overlay problem with segmented designs


% Restrictions of the Segmented design ( da kann ich auch allgemeine Punkte zu segmentierten
% speicherdesigns aus grundlagendpapern rausnehmen)

%Flexibility limitation -> flexibility of the virtual memory system
% is still restricted by the tlb structure
% -> other approach jacobSoftwaremanagedAddressTranslation1997

% More fine grained controll over TLBs via CSRs -> Inspo mips ( Evaluation of Implementation, not necesarrily design)

\paragraph*{Fragmentation}
%Processes have fixed size of memory and may not use all of it
\paragraph*{Limited Process count}
%Maybe appropiate for embedded applications?

% -------------------------------------------------------------------------------------------------
\subsection{Memory System Requirements} % Shortcommings of my implementation
Für eine qualitative Analyse des implementierten Virtual Memory Systems bietet \cite{jacobSoftwaremanagedAddressTranslation1997}
eine gute Vorlage. Dort wurden die zentralen Anforderungen an ein Virtual Memory System auf Grundlage
von vielen gängigen microarchitekturen und Betriebssystemen herausgearbeitet.


%Kein Demand Paging, Kein swapping so möglich
% Was ist mit dem paging passiert -> Automatisches Tauschen von Seiten zwischen Haupt und Nebenspeicher?
% In xv6 nicht implementiert, könnte es mit dem Segmentierten Design implementiert werden

% TODO -> Perspektive richtung Embedded, wo isolation gewünscht aber kein Nebenspeicher existiert
% Idee -> Address spaces with different sizes, so that slots can be swapped, or address spaces can be merged
% Memory hungry process can get multiple Address spaces

%
% -------------------------------------------------------------------------------------------------
\section{Quantitative Analysis}
\subsection{Problems with measuring performance of an emulated system}
% Warum geht es nicht -> Emuliert

% Ohne caches -> Emualtion vs Real Hardware -> Warum keine quantitaive analyse

% -------------------------------------------------------------------------------------------------
% DISCUSSION ON REAL HARDWARE
% Extensibility foundations: CSRs, satp mode field open slots for more memory modes
%   Configurable HASHING MMU
% \subsubsection{Switching between HW and SW TLB miss handling}

% -------------------------------------------------------------------------------------------------



%?????


% Vergleich MIPS Page table fast path

% Inside L4 Mips fastpath vs slowpath




% Hardware erweiterung wenn die funktion parametrisierbar ist




