\chapter{Conclusion}

\label{chap:conclusion}

\cite{choudhuri2005software} % Potential use case?


% CONCLUSION: IT WORKS
This paper presented the theory behind and the modifications of the xv6 operating system running on the QEMU/RISC-V platform to transform the fundamental control flow of the virtual memory path. The result is a simple platform is a foundation to start experimenting with mapping functions that replace the typical page table structure used by commodity hardware.

The paper also presented a mapping function implemented on that platform, realizing a ( quite restricted ) virtual memory system, that works completely without using any bookkeeping of mappings. The changes are completely transparent to the system call interface of the operating system and do thus not require changes to programs running on the xv6 operating system.

However, the evaluation showed clear deficits when compared to a conventional VM system regarding functional requirements to a memory system. Also, the mapping function has not been evaluated quantitatively and might suffer from performance issues on specific workloads. Also, the general problems with software-managed VM systems still persist in the mapping function design.

As this thesis provides a platform for further testing with function based VM systems, more work might go into further research on how these functions could look like and how the functional requirements to VM systems could be realized with this approach. If some functions prove effective at providing proper virtual memory support, hardware designs to implement such functions based on simple arithmetic operations may be examined. Hardware support for VM mapping functions would alleviate costs associated with context switches and pipeline disruptions.

Exploration on good functions for memory mapping could start by looking into hash functions used by current inverted paging approaches, which already show attributes very useful for distributing pages across a smaller physical space while keeping spatial locality and combating collisions.