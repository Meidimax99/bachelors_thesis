\chapter{Introduction} % Main chapter title
% TODO CHECK 2-3 Seiten
% TODO CHECK Keine Ergebnisse
% TODO CHECK Keine Definitionen
% TODO CHECK Alles relevante vorhanden

\label{Intro}

\todo{CITE Architectural and operating system support for virtual memory}
Virtual memory is used by computer systems at all scales. Everything from systems running in huge data
centers to embedded computers rely on virtual memory.



% Relevance: Most computers today use a page-based virtual memory approach
% Very short summarization of the way page-based virtual memory works, ending with the amount of memory references
% barrTranslationCachingSkip -> memory references keep on increasing as memory requirements increase

% Soruce: Virtual memory: issues of implementation

\begin{itemize}
    \item Software Managed TLB
    \item Not based on page tables but a simple function to reduce memory accesses to a minimum
\end{itemize}


The RISC-V Instruction Set manual warns that ''[...]Software TLB refills are a performance bottleneck[...]''\cite{RISCVInstructionSet},
but also suggests that they may be realized by implementing a machine-mode trap handler as an extension to the
most-priviledged machine mode. This paper will explore just that.
% Auch -> Proof of concept
% Und: Keine Page table walks
% Discussion: -> A look at several ...
\cite{jacob1998look}

% TODO How do inverted paging schemes compare?
% -> virtual memory: Issues of impl
% -> A Look at Several Memory Management Units, TLB-Refill Mechanisms, and Page Table Organizations

% TODO Very short: How the TLB is used

% TODO Problem Statement: Are paging structures even necesarry or can virtual memory purely be realized
%       by a simple software-defined function that refills the tlb?

% Relevanz: Virtual memory in embedded devices -> segmented design may suffice
% -> Reference?


% TODO Forschungsfrage
% TODO Inhaltliche Abgrenzung und Ausschließen von Teilproblemen
% TODO Aufbau der Arbeit
% TODO Relevanz
% TODO Geschichte/Einführende Zitate
% \section{History of Virtual Memory}
% \subsection{Atlas}
% \subsection{Overlaying}

%Quick Facts
% Intel 5 Level Paging intelTLBsPagingStructureCaches2008
% Sv57 RISC-V Paging RISCVInstructionSet


% WHAT DOES THIS PAPER DO
% -> SW TLB Handling Proof Of concept on riscv
% -> Use that TLB handling to realize a Segmented vm design not using any page tables