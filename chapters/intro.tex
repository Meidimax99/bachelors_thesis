\chapter{Introduction} % Main chapter title
% TODO CHECK 2-3 Seiten
% TODO CHECK Keine Ergebnisse
% TODO CHECK Keine Definitionen
% TODO CHECK Alles relevante vorhanden



% -------------------------------------------------------------------------------------------------
%                                             Structure
% -------------------------------------------------------------------------------------------------



% Motivation for VM
% Importance of vm
% Functional requirements
% Description of Problems with VM
% Unclear interfaces
% Performance -> Applications with poor spatial locality
% Description of Approach

% Proof of concept
% Description of Contents









% -------------------------------------------------------------------------------------------------
% -------------------------------------------------------------------------------------------------
% -------------------------------------------------------------------------------------------------
% -------------------------------------------------------------------------------------------------
% Motivation
% Most Computer use Virtual Memory

% THIS SECTION may primarily be about some facts about VM

Virtual memory provides a multitude of features to applications that vastly simplify the life of application programmers \cite{jacob1998virtualissues}. Computer systems of all scales, ranging from small embedded devices to huge data centers use virtual memory \cite{bhattacharjee2017architectural}.\\
While originally being used to automate the task of swapping pages of processes between main memory and secondary storage transparently to the processes \cite{jacob1998virtualissues}, it now is the foundation for security, reliability, process isolation and flexibility \cite{wales1999virtual,jacobVirtualMemoryContemporary1998}.

With ever increasing memory sizes of systems and memory requirements of applications, the overhead of virtual memory systems, being developed for systems that only had scarce resources available \cite{halbuer2023morsels}, is degrading performance and increase power consumption significantly \cite{zagieboylo2020cost}.\\
Orthodox hierarchical page table organizations \cite{tanenbaumOS} can get as deep as 5 levels in commodity hardware \cite{intel5LevelPaging5Level2017}. Not only does this add a level of indirection for each page table walk, this penalty is even higher in virtualized systems, potentially accounting for up to 5\% - 90\% of the runtime \cite{yaniv2016hash}.

Alternative designs like inverted page tables realize page table mappings using hash functions \cite{tanenbaumOS}. While reducing the number of additional memory references, these approaches have problems on their own: Some features of virtual memory systems become harder to implement, exacting additional performance penalties \cite{yaniv2016hash} and page table lookups can get a lot more expense when following collision chains \cite{jacob1998look}.

The Related Work chapter \ref{chap:related} will show, that there are a lot of different angels to optimize the virtual memory system.\\
This shows that there is little agreement on how the virtual memory system is best implemented \cite{jacob1998look}.\\
%Quick Facts
% Intel 5 Level Paging intelTLBsPagingStructureCaches2008
% Sv57 RISC-V Paging RISCVInstructionSet


% Plethora of interfaces suggesting performance problems

% [ A look at several ]
% implemented on either the hardware or software side of the interface. The myriad of design choices and incompatible hardware mechanisms suggests potential performance problems, especially since increasing numbers of systems (even embedded systems) are using memory management. A comparative study of the implementation choices in virtual memory should therefore aid system-level designers.
\label{Intro}

\todo{CITE Architectural and operating system support for virtual memory}
Virtual memory is used by computer systems at all scales. Everything from systems running in huge data
centers to embedded computers rely on virtual memory.

% Relevance: Most computers today use a page-based virtual memory approach
Current radix page tables can get up to 5 levels deep \cite{intel5LevelPaging5Level2017}. Every level
adds another memory access to the page table lookup.\\
Caching structures can alleviate the cost by storing complete or partial translation results \cite{van2002memory}.
But especially large-memory applications, virtualized environments and graph-processing applications
suffer extensively from address translation costs \cite{zagieboylo2020cost}.


% -------------------------------------------------------------------------------------------------
% Problems with VM

% Many different interfaces
\cite{jacobVirtualMemoryContemporary1998,jacob1998virtualissues}
% SOME NUMBERS PLEASE -> [ A look at several ...] abstract
% Previous approaches aimed to reduce the memory footprint or optimize cache accesses, or reduce page table
% walks to a minimum, this approach aims to eliminate paging structures completely

% Very short summarization of the way page-based virtual memory works, ending with the amount of memory references
% barrTranslationCachingSkip -> memory references keep on increasing as memory requirements increase

While previous work tried to either reduce the number of memory accesses \todo{CITE Lidtke ,Skip no walk, single hit uppsala paper},
decrease the occurrence of a TLB miss and decrease the latency of handling a TLB Miss, this paper
presents an approach to create a virtual memory scheme not requiring paging structures at all.\\


% TODO Problem Statement: Are paging structures even necesarry or can virtual memory purely be realized
%       by a simple software-defined function that refills the tlb?
%

% Conclusion on previous Work -> Still need a memory access

% The Idea -> Getting rid of any memory access in favor of simple arithmetic operation (eg. hash functions)
% statefull -> stateless

% -------------------------------------------------------------------------------------------------
% -------------------------------------------------------------------------------------------------
% Description of Approach
% TODO Very short: How the TLB is used

% The approach is not directly the TLB filling, thats just one
% implementation of the process.

% The main idea is to replace page table walks with simple function calls

Current hardware-supported virtual memory systems suffer from inflexibility and costly memory accesses \cite{jacob1998look}.\\
Software-based approaches are a lot more flexible, but are, so far, based on page table lookups as well.

% Soruce: Virtual memory: issues of implementation

\begin{itemize}
    \item Software Managed TLB
    \item Not based on page tables but a simple function to reduce memory accesses to a minimum
\end{itemize}


The RISC-V Instruction Set manual warns that ''[...]Software TLB refills are a performance bottleneck[...]''\cite{RISCVInstructionSet},
but also suggests that they may be realized by implementing a machine-mode trap handler as an extension to the
most-priviledged machine mode. This paper will explore just that.
% Auch -> Proof of concept
% Und: Keine Page table walks
% Discussion: -> A look at several ...


% WHAT DOES THIS PAPER DO
% -> SW TLB Handling Proof Of concept on riscv
% -> Use that TLB handling to realize a Segmented vm design not using any page tables


% -------------------------------------------------------------------------------------------------
% -------------------------------------------------------------------------------------------------
% Proof of concept


% -------------------------------------------------------------------------------------------------
% -------------------------------------------------------------------------------------------------
% Description of Contents


% Inhaltsbeschreibung
Chapter 2 provides an overview of the fundamentals of virtual memory systems, software-managed virtual memory systems and some basics of the virtual memory system of the RISC-V ISA.\\
This forms the fundament for the following chapters.

Chapter 3 takes a look at previous work that also approaches the topic of optimizing virtual memory systems.

Chapter 4 describes the theoretical development of the idea presented in this paper and thus provides
the foundation for the implementation.\\

The implementation is described in Chapter 5. It delves into the specifics of the programming platforms
and outlines the implementation process step by step. This chapter also includes an overview of the debugging
techniques used for verification and troubleshooting of the implementation.\\

Chapter 6 critically examines the current state of the theoretical development and implementation,
analyzing the results in light of typical requirements for other virtual memory systems.
It also includes a discussion on further deepening the approach.\\
\todo{future work yes or no?}
In Chapter 7, suggestions are made for future work building upon this thesis. These suggestions refer
to previously excluded topics, such as hardware optimizations of the idea.

The eighth and final chapter provides a summary of the paper and its results.\\


% -------------------------------------------------------------------------------------------------






\cite{zagieboylo2020cost} has nice numbers in introduction for cost of VM

% -------------------------------------------------------------------------------------------------
% -------------------------------------------------------------------------------------------------
% -------------------------------------------------------------------------------------------------
% -------------------------------------------------------------------------------------------------
