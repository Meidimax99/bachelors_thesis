\chapter{Introduction} % Main chapter title
% TODO CHECK 2-3 Seiten
% TODO CHECK Keine Ergebnisse
% TODO CHECK Keine Definitionen
% TODO CHECK Alles relevante vorhanden



% -------------------------------------------------------------------------------------------------
%                                             Structure
% -------------------------------------------------------------------------------------------------



% Motivation for VM
% Importance of vm
% Functional requirements
% Description of Problems with VM
% Unclear interfaces
% Performance -> Applications with poor spatial locality
% Description of Approach

% Proof of concept
% Description of Contents









% -------------------------------------------------------------------------------------------------
% -------------------------------------------------------------------------------------------------
% -------------------------------------------------------------------------------------------------
% -------------------------------------------------------------------------------------------------
% Motivation
% Most Computer use Virtual Memory

% THIS SECTION may primarily be about some facts about VM

\textbf{Virtual memory} provides a multitude of features which vastly simplify the life of application programmers \cite{jacob1998virtualissues}. Computer systems of all scales, ranging from small embedded devices to huge data centers use virtual memory \cite{bhattacharjee2017architectural}.
While originally being used to automate the task of swapping pages of processes between main memory and secondary storage transparently to the processes \cite{jacob1998virtualissues}, it now is the foundation of security, reliability, process isolation and flexibility \cite{wales1999virtual,jacobVirtualMemoryContemporary1998}.

With ever increasing memory sizes of systems and memory requirements of applications, the overhead of virtual memory systems, being developed for systems that only had scarce resources available \cite{halbuer2023morsels}, degrades performance and significantly increases power consumption \cite{zagieboylo2020cost}.
Orthodox hierarchical page table organizations \cite{tanenbaumOS} can get as deep as 5 levels in commodity hardware \cite{intel5LevelPaging5Level2017}. Not only does this add a level of indirection for each page table walk, this cost is even higher in virtualized systems, potentially accounting for up to 5\% - 90\% of the runtime \cite{yaniv2016hash}.

\textbf{Alternative designs} like inverted page tables realize page table mappings using hash functions \cite{tanenbaumOS}. While reducing the number of additional memory references, these approaches have problems on their own: Some features of virtual memory systems become harder to implement, exacting additional performance penalties \cite{yaniv2016hash} and page table lookups become a lot more expensive when following collision chains \cite{jacob1998look}.

Chapter \ref{chap:related} will show, that there are a lot of different approaches to optimize the virtual memory system.
This shows that there is little agreement on how the virtual memory system is best implemented \cite{jacob1998look}.
However, most of the optimization approaches still rely on page table structures to do the bookkeeping of the mappings.

This thesis explores the idea of getting rid of page table structures all together and base the mapping of virtual to physical addresses on specially crafted functions (for example non-cryptographic hash functions \cite{mittelbach2021non}).
Hash functions implemented by simple arithmetic instructions are orders of magnitude faster than the multiple memory accesses required by a page table walk \cite{tanenbaumOS}. Allowing them to be defined in software gives the operating system a lot of flexibility to fit the implementation of the mapping function to the custom needs of applications.

This thesis will describe the theory and implementation of a platform that facilitates the definition and the experimentation with such mapping functions and shows a first simple mapping function that realizes a simplified virtual memory system without any page tables.


% Inhaltsbeschreibung
\textbf{Chapter 2} provides an overview of the fundamentals of virtual memory systems, software-managed virtual memory systems and some basics of the virtual memory system of the RISC-V ISA.

\textbf{Chapter 3} takes a look at previous work which approaches the topic of optimizing virtual memory systems as well.

\textbf{Chapter 4} describes the theoretical development of the idea presented in this paper and thus provides the foundation for the implementation.

\textbf{Chapter 5} describes the implementation. It delves into the specifics of the programming platforms and outlines the implementation process step by step. This chapter also includes an overview of the debugging techniques used for verification and troubleshooting of the implementation.

\textbf{Chapter 6} critically examines the current state of the theoretical development and implementation, analyzing the results in light of typical requirements for other virtual memory systems. It also includes a discussion on further deepening the approach.

% In \textbf{Chapter 7}, suggestions are made for future work building upon this thesis. These suggestions refer to previously excluded topics, such as hardware optimizations of the idea.



% -------------------------------------------------------------------------------------------------







% -------------------------------------------------------------------------------------------------
% -------------------------------------------------------------------------------------------------
% -------------------------------------------------------------------------------------------------
% -------------------------------------------------------------------------------------------------
