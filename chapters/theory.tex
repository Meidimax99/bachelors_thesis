\chapter{Theory}

\label{chap:theory}

\todo{elaboration on xv6 here or in fundamentals (or an extra platform chapter)?? -> because theory
    may be independent from the platform in big parts, but still needs to be taken into consideration (like with
    csrs)}


%elaboration on general costs of page table walks -> basically exactly what related
% work motivated their designs with

% Previous approaches aimed to reduce the memory footprint or optimize cache accesses, or reduce page table
% walks to a minimum, this approach aims to eliminate paging structures completely
While previous work tried to either reduce the number of memory accesses \todo{CITE Lidtke ,Skip no walk, single hit uppsala paper},
decrease the occurrence of a TLB miss and decrease the latency of handling a TLB Miss, this paper
presents an approach to create a virtual memory scheme not requiring paging structures at all.\\

Not unlike the software-managed address translation design presented by Jacob and Mudge \cite{jacobSoftwaremanagedAddressTranslation1997},
the \textit{softtlb} approach utilizes a exception triggered by a cache miss to invoke a software-defined
exception handler.\\
Unlike the \textit{softvm} design, this approach will be based on handling an exception that is triggered
when die TLB misses \todo{TLB and its role explained?}.
The handler code for this \textit{TLB\_Miss} exception will from now on be called \textit{TLB Manager}.

The \textit{TLB Manager}s job is then to resolve the exception by filling the TLB for the failing address
that triggered the exception.

...

\section{Choosing a platform}
Choosing a platform for designing and implementing the \textit{softtlb} approach is a non-linear tradeoff between
ease of use and suitability.\\
A obvious choice for a workbench to design and implement \textit{softtlb} for would be MIPS/L4 \cite{heiserAnatomyHighPerformanceMicrokernel}
% todo why?


\section{Triggering a \textit{TLB\_Miss} exception}
% \todo{Maybe motivation for VM on specific platforms like embedded platforms}
% \section{Problem Statement}
% %Reiteration of Problem Statement in introduction
% \subsection{Der Trade-off von Hardware supported Virual Memory} % This is motivation
% \section{Idea}


% \section{L4/MIPS als Vorlage für Software TLB Management in RISCV (Qemu)}
% \subsection{TLB Miss Exception}
% \subsection{TLB Loading instructions}
% \paragraph{tlbwi}
% \paragraph{tlbwr}
% Comparison to MIPS tlb software

\section{Writing TLB entries}
\subsection{RISC-V CSRs}
RISC-V and its extensions currently provide no support for modifying the TLB in software.
RISC-V does however provide a lot of extensibility with the \texttt{Control and Status Registers} (CSRs).\\
CSRs are part of the RISC-V priviledged architecture and are provided by the \texttt{Zicsr} extension\cite{RISCVInstructionSet}.\\

% Section describing RISC-V CSRs
\todo{table of csr space in appendix?}
The RISC-V ISA provides a 12-bit encoding space for 4096 CSRs. A CSR address is logically split
into four parts: The top two bits \texttt{csr[11:10]} specify whether the CSR is read/write or read-only;
\texttt{csr[9:8]} encode the minimum priviledge level that is allowed to access the CSR; \texttt{csr[7:4]}
may be partially used to define a specific use for a range of CSRs. E.g. CSRs with an address
between \texttt{0x7B0} and \texttt{0x7BF} shall be used for Debug-mode-only CSRs. The format
of the CSR addresses is also depicted in figure \ref{fig:theory:csr}.

% RISC-V CSR address bytefield
\begin{figure*}[t]
    \centering
    \begin{bytefield}[bitwidth={2em}, bitformatting={\bfseries}, boxformatting={\centering}]{12}
        \bitheader[endianness=big]{11,10,9,8,7,4,3,0} \\
        \bitbox{2}{RW/RO} &
        \bitbox{2}{Priv} &
        \bitbox{4}{Useage} &
        \bitbox{4}{Index}
    \end{bytefield}
    \caption[RISC-V CSR address format]{RISC-V CSR address format}
    \label{fig:theory:csr}
\end{figure*}

Each legal CSR address identifies a CSR. The size of the CSRs identified by the CSR address depends
on the values of the SXLEN and UXLEN fields in the \textbf{mstatus} register. Currently, the
specification \cite{RISCVInstructionSet} allows for 32 bit, 64 bit and 128 bit.\\ \todo{disclaimer which we will be using here?}
% Short elaboration on using the csr to write TLB entries -> in the end its on the hardware implementor
RISC-V provides dedicated instructions for read/write and bit manipulation with both register values
and immediates.
\todo{ hardware overhead for writing TLB with CSRs -> future work, is it worth it??}

\subsection{CSR format}
% What format, how big, how many csrs?
Now with CSRs we have a mechanism to add custom behaviour to the ISA. That answers the ''How?''
of writing TLB entries in software.\\
The next step is to figure out what data is required to create a TLB entry and in what format
this data is best communicated via the CSRs to the computer.\\

% Inspiration: MIPS
\subsection{MIPS as an inspiration for TLB writes}
MIPS is a good source of inspiration for a possible CSR format design, since it already provides
instructions to modify the TLB.\\
The MIPS64 instruction set manual \cite{MIPS_Architecture_MIPS64_InstructionSet_AFP_P_MD00087_06052016}
shows a number of different instructions concerned with the invalidation, probing, flushing, reading
and writing (indexed and random).\\
The most interesting for a first design would be the \texttt{TLBWR} instruction for writing a TLB
entry at a random index. With a similar instruction in RISC-V, we can already implement a purely
software-controlled virtual memory system.\\
The other types of TLB instructions that MIPS provides are not strictly necesarry,
except for flushing. Without being able to flush existing translations from the TLB,
user mode processes may try to access physical mappings stemming from other processes.\todo{can this even happen? -> why else would we need to flush the tlb??}
But the RISC-V priviledged Architecture already provides this functionality
with the \texttt{sfence.vma} instruction \cite{riscvreader}.

\paragraph{MIPS - TLBWR} The arguments for the instruction need to be written in some
other registers - \texttt{EntryHi}, \texttt{EntryLo0}, \texttt{EntryLo1} and \texttt{PageMask}.

% TODO ORDER!
%MIPS TLB Structure


% Compare my implementation to a mips one -> Heiser TLB Fastpath
% Fixed TLB format, but flexibilty of CSR format

% Qemu TLB

% Typical TLB entry format

% selective TLB flushing -> RISCV ASIDs (or even no TLB flushing  because VAs contain ASID?)

% How is the RISCV Tlb indexed -> Check Qemu source

% Replacement policy
%   MIPS -> Indexed writes
A advantage of software-managed TLBs is, that the operating system can implement custom
TLB replacement policies, that may even change depending on workload, programs running
and other circumstances.\\
The default replacement strategy for the MIPS \texttt{tlbwr} instruction is to simply
use the value of the \texttt{C0\_RANDOM} register as the index for the next TLB entry
to be replaced. The name of that register is missleading, because it is not actually a
random value, but it is rather decremented on each instruction\cite{heiserAnatomyHighPerformanceMicrokernel}.\\
It is not clear whether this is a sensible replacement strategy,
but it can be used to ensure that the same TLB slot is not used for every \texttt{tlbwr} if the implementation
does not provide for any further replacement strategy.\\
Some TLB entries can be protecte from this ''random'' replacement by setting a value in the \texttt{C0\_WIRED}
register. The value in this register represents a lower bound, protecting all TLB entries that lie below it.\\
This is useful to keep some mappings in the TLB that are valid all the time.

% TODO should this be here? We should probably have a discussion about xv6 first
%Protected TLB entries in xv6 -> Trampoline page
Having a protected space of TLB entries can especially be useful for global mappings. xv6 employs such a global
mapping for every process and the kernel with the trampoline page.\todo{trampoline page should be explained in fundamentals/platform}

% Future work -> finer control

% \section{Motivation}
% \subsection{Trade-offs of traditional Virtual Memory Approaches}

% Problems
% Context switch to handler -> Pipeline flush, reorder buffer flushed
%   Possible solution in In-Line Interrupt Handling for Software-Managed TLBs


% Segmented design: Problems and comparison to contemporary virtual memory systems
% -> REF Liedtke: On the realization of huge sparsely-occupied and fine-grained address spaces

% Riscv can flush entries of a specific ASID only, this means that it keeps that information
% Somewhere in the TLB structure


% xv6 

% Page size (tlb reach?)

% Qemu -> TLB structure, replacement
