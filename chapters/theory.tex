\chapter{Theory}

\label{chap:theory}

% Previous approaches aimed to reduce the memory footprint or optimize cache accesses, or reduce page table
% walks to a minimum, this approach aims to eliminate paging structures completely
While previous work tried to either reduce the number of memory accesses \todo{CITE Lidtke ,Skip no walk, single hit uppsala paper},
decrease the occurrence of a TLB miss and decrease the latency of handling a TLB Miss, this paper
presents an approach to create a virtual memory scheme not requiring paging structures at all.\\

Not unlike the software-managed address translation design presented by Jacob and Mudge \cite{jacobSoftwaremanagedAddressTranslation1997},
the \textit{softtlb} approach utilizes a exception triggered by a cache miss to invoke a software-defined
exception handler.\\
Unlike the \textit{softvm} design, this approach will be based on handling an exception that is triggered
when die TLB misses \todo{TLB and its role explained?}.
The handler code for this \textit{TLB\_Miss} exception will from now on be called \textit{TLB Manager}.

The \textit{TLB Manager}s job is then to resolve the exception by filling the TLB for the failing address
that triggered the exception.

...

\section{Choosing a platform}
Choosing a platform for designing and implementing the \textit{softtlb} approach is a non-linear tradeoff between
ease of use and suitability.\\
A obvious choice for a workbench to design and implement \textit{softtlb} for would be MIPS/L4 \cite{heiserAnatomyHighPerformanceMicrokernel}

\section{Triggering a \textit{TLB\_Miss} exception}
% \todo{Maybe motivation for VM on specific platforms like embedded platforms}
% \section{Problem Statement}
% %Reiteration of Problem Statement in introduction
% \subsection{Der Trade-off von Hardware supported Virual Memory} % This is motivation
% \section{Idea}


% \section{L4/MIPS als Vorlage für Software TLB Management in RISCV (Qemu)}
% \subsection{TLB Miss Exception}
% \subsection{TLB Loading instructions}
% \paragraph{tlbwi}
% \paragraph{tlbwr}
% Comparison to MIPS tlb software



% \section{Motivation}
% \subsection{Trade-offs of traditional Virtual Memory Approaches}

% Problems
% Context switch to handler -> Pipeline flush, reorder buffer flushed
%   Possible solution in In-Line Interrupt Handling for Software-Managed TLBs


% Segmented design: Problems and comparison to contemporary virtual memory systems
% -> REF Liedtke: On the realization of huge sparsely-occupied and fine-grained address spaces

