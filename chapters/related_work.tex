\chapter{Related Work}

\label{chap:related}

% \section{Liedtke: Guarded Page Tables}
% \todo{Should this be in the fundamentals chapter? Inverted Page tables are used relatively much}
% \section{Inverted Page Tables}

% \section{Other software-based TLB miss handlers / Page table walkers?}

\cite{jacobSoftwaremanagedAddressTranslation1997} explores software-managed
address translation and analyses the efficiency of a PowerPC implementation
of the presented design they call \textit{softvm}. They show that
software-managed address translation can achieve better performance and
at the same time simplify hardware by dispensing with translation caches and
the hardware state-machine for walking the page table.\\
The approach is based on handling virtually indexed and tagged cache misses in
software.
With sufficiently-sized virtual caches the system can go for long periods
without requiring translations.
Not unlike this paper, they extend the PowerPC architecture by two new instructions
to write entries to the cache. However, their design uses software page table
walks to find translations on cache-miss, while the approach presented in this
paper presents a software-based TLB fill mechanism with segmented
memory allocation.\\\\
% TODO Further elaboration on the Background chapter of jacobSoftwaremanagedAddressTranslation1997


\cite{barrTranslationCachingSkip}

% Every walk’s a hit: making page walks single-access cache hits

% Liedtke GPTs + Heise Implementation -> Software Based approach, but still inherently uses a page table
%TODO split in different areas
% Software Managed Virtual Memory
% Hardware optimizations -> Translation caches
% software optimizations
% Alternative page table designs

% Related work exploring Other caches, like Page Tables in L2 -> Out of scope for this work


% TODO Mips