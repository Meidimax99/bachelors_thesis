\chapter{Related Work}

\label{chap:related}
Diese Arbeit lässt sich grob in den Raum von Virtual Memory Optimierung einordnen.
Es gibt eine große Menge an Literatur die sich mit der der Optimierung des Virtual Memory Systems
beschäftigt. Dieses liegt schließlich auf dem kritischen Pfad jeder Memory Operation.\\
Dabei gibt es verschiedene Ansätze oder Winkel um das System anzugehen:
\todo{Diese Punkte müssen noch zusammengefasst werden}
\begin{itemize}
    \item Die Pagetablestruktur, z.B. invertiert oder hierarchisch
    \item Caching von Seiten, PTEs im TLB und indizierung von Caches
    \item Replacementstrategien der Caches
    \item Cachegrößen, Assoziativität
    \item translation caches in MMU (not TLB), Page Walk caching
    \item TODO weitere Aspekte
    \item software optimizations
    \item Hardware optimazations, hardware support, flexibility
\end{itemize}

% -------------------------------------------------------------------------------------------------

% Guarded Page Tables
\textbf{\cite{liedtkeGPT}}

% -------------------------------------------------------------------------------------------------

% MIPS GPT Impl
\textbf{\cite{heiserAnatomyHighPerformanceMicrokernel}}

% -------------------------------------------------------------------------------------------------

% In Cache Software Managed Address Translation
\textbf{\cite{jacobSoftwaremanagedAddressTranslation1997}}explores software-managedaddress translation and analyses the efficiency of a PowerPC implementation
of the presented design they call \textit{softvm}. They show that
software-managed address translation can achieve better performance and
at the same time simplify hardware by dispensing with translation caches and
the hardware state-machine for walking the page table.\\
The approach is based on handling virtually indexed and tagged cache misses in
software.
With sufficiently-sized virtual caches the system can go for long periods
without requiring translations.
Not unlike this paper, they extend the PowerPC architecture by two new instructions
to write entries to the cache. However, their design uses software page table
walks to find translations on cache-miss, while the approach presented in this
paper presents a software-based TLB fill mechanism with segmented
memory allocation.\\
% TODO Further elaboration on the Background chapter of jacobSoftwaremanagedAddressTranslation1997

% -------------------------------------------------------------------------------------------------

% Translation Caching skip dont walk -> Translation Caches
\textbf{\cite{barrTranslationCachingSkip}}betrachtet den Design Space der Translation Caches in MMUs. Diese Caches sind nicht mit den TLBs
vergleichbar, die ganze übersetzungen von virtuellen zu PTEs enthalten. Die Translation Caches
die hier betrachtet werden enthalten partielle Übersetzungen, die es bei einem Cache Hit möglich
machen einzelne Schritte beim durchschreiten des Page Table Baumes zu überspringen und somit
einen oder mehrere Speicherzugriffe einzusparen. Sonst würde die Übersetzung pro Level eine
Speicherreferenz benötigen. Diese Caches werden auch Page Walk Caches (PWC) \cite{yaniv2016hash} genannt.\\
Betrachtet wird das konkrete Translation Cache Design von AMD und Intel plattformen und vergleichend
mit 3 weiteren von den Authoren vorgeschlagenen Designs präsentiert. Barr et al. schließen,
dass die Radix Page tables durch cachen von Einträgen auf hohen Page Table levels invertierte Page
Table designs autperformen können.\\
Diese Arbeit beschäftigt sich mit der Optimierung des Memory Pfades durch abschaffen der Page Tables
und durch einen Software-Kontrollierten TLB und zieht Caches, abgesehen von dem TLB nicht weiter in
betracht.

% -------------------------------------------------------------------------------------------------

% Hash dont Cache
\textbf{\cite{yaniv2016hash}}fordert die Ergebnise von \cite{barrTranslationCachingSkip} heraus und argumentiert, dass die erzielten
Ergebnisse auf einer subobtimalen implementation der invertierten Page table basieren.\\
Sie schließen, dass eine gut optimierte Invertierte Page Table auch eine mit PWCs ausgestattete
Radix Page Table outperformed. Sie gehen jedoch auch auf die conceptuellen nachteile von Invertierten
Page tables ein, z.B. ist es schwieriger Super Pages zu implementieren.\\
Die Arbeit beschäftigt sich näher mir den unterschieden verschiedener Page Table designs und auch mit
den Requirements an ein Memory system (wie zum Beispiel superpages oder page sharing). Diese Requirements sind auch
für das hier vorgestellte Design wichtig. Sonst zielt diese Arbeit aber eher darauf ab keine
Page Table strukturen zu verwenden.



% -------------------------------------------------------------------------------------------------

% Every walk’s a hit: making page walks single-access cache hits
\textbf{\cite{park2022every}} idenzifiziert, dass die heutigen Speicherkapazitäten die TLB Coverage weiter und weiter übertreffen
und somit speicherhungrige Applikationen unter vielen Page Table Walks (PTWs) leiden.\\
Es werden zwei Ansätze vorgestellt um die damit verbundenen Kosten zu reduzieren:
Der erste Ansatz zielt darauf ab die anzahl der Speicherreferenzen pro PTW zu reduzieren. Dabei werden
zwei Level der Page Table in ein Level verbunden.
Der zweite Ansatz ändert die Cache replacement policy so ab, dass Cache Einträge die PTEs enthalten
bei Perioden mit vielen TLB misses eher im Cache behalten werden, so dass die PTWs direkt über
den Cache laufen können und nicht erst aus dem Hauptspeicher geladen werden müssen.\\
Sie zeigen eine 2.3\% performance Verbesserung mit dem der abflachung des Page Table Trees,
6.2\% durch die Cache Priorisierung und combiniert eine verbesserung der Performance um 9.2\%.
Beide Ansätze zielen auf eine Optimierung des Zugriffes auf die Page Table Strukturen hin und sind
damit getrennt von dieser Arbeit zu sehen, da diese Arbeit darauf abzielt, ohne diese Strukturen
auszukommen.


% -------------------------------------------------------------------------------------------------

% Elastic Cuckoo Page Table
\textbf{\cite{skarlatos2020elastic}} presents a novel page table design called \textit{Elastic Cuckoo Page
    Tables}. Das Design nutzt den Memory-level Parallelismus aus um Page Tabel lookups komplett parallelisiert
anstellen zu können. At the core of the design lies the Elastic Cuckoo Hashing algorithm witch allows
multiple hashing locations for a given element and also allows the hash table to be efficitenly gradually
resized. Skarlatos et all. show an application execution speed up of 3-18\% using the Elastic Cuckoo
Page Table design.\\



The related work presented here focuses on optimizing current page table designs and their hardware
support. This work will explore the feasibility of implementing virtual memory without any page table
whatsoever. % How? -> Theory!