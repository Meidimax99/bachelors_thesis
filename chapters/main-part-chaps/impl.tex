\chapter{Implementation}

\label{chap:impl}

% \section{Platform}
% \subsection{Platform Considerations}
% % Which aspects are important for chosing the platform for this project?
% % Why have I not chosen L4, Mips, etc? They already have some preconditions that are pretty good for my work (software handler, tlb instructions)
% \subsection{xv6}
% \subsection{RISC-V}
% \subsection{Qemu}
% \paragraph{softmmu Target}




\section{Relevant Modules in the xv6 Source}
\paragraph*{kalloc - physical memory allocator}
\paragraph*{proc}
\paragraph*{vm.c}
\subsection*{Syscalls to change}
% System call interface should stay the same so that existing processes can continue
% as usual -> new virtual memory scheme should be completely transparent
\paragraph*{fork}
% creating a new addresse space -> fixed number of processes ? ?
\paragraph*{exit}
% freeing memory?
\paragraph*{exec}
% same as fork?
\paragraph*{sbrk}
% allocating new memory
Sbrk is the system call with which a program can change its program break. The program break
is the end of the programs data segment. This system call is typically used to either
increase or decrease a programs memory.
Sbrk would typically be called by the memory manager of the user program. Calls
to malloc would for example call sbrk when there is not enough memory available to
satisfy the call to malloc.
Internally, sbrk calls a function called growproc(n) which grows or shrinks to calling
processes memory by n bytes.
growproc will then call either uvmalloc() or uvmdealloc(). These functions need to be
changed to use the new allocation scheme.

\paragraph*{Experimental Allocation Scheme}
xv6 keeps all of its unused physical memory in a linked list. The linked list itself is stored
in the free blocks, as they are not used for anything else. When a process asks for more memory
the kalloc() function retrieves the first element from the linked list, updates the head
of the list to point to the next element and returns the pointer to the first element,
which is the same as the physical address for the page managed by that first list item.
kalloc() is called by the uvmalloc() function, and since we want to change the way
the virtual memory management works, it would make sense to directly change the uvmalloc() function.
However, in order to properly test the physical memory allocation first, we will change
the allocation scheme first and then the mapping, as the trampoline page would not work anymore.

The new allocation scheme statically assigns a fixed portion of the available physical memory
to each process. The size is determined by the size of physical memory and the maximum number
of processes.
kalloc() will then use the process id, the current size of the process and the requested
increase in size to determine the next page for the process.

\subsection*{Why does this not work for anything else but the sbrk syscall}
- kalloc with sbrk uses the current process size to determine the next free space in
the address space of the process
- The processes size only keeps track of the text data and so on segments. So only stuff
the program is aware of. Page table pages need to be allocated per process as well but
do not contribute to the overall size of the process so they can not be put into the
processes memory space. This has the direct advantage that users can not access their page tables
but this was never the problem because pages containing page tables could simply be marked
as not user accessible. And of course this brings some disadvantages: Kernel size becomes
a scarce resource and user address spaces are not self contained anymore, page tables now reside
somewhere else entirely.
But since it is the goal to get rid of page tables anyways, its not too big of a problem.
When we get rid of the page tables, we get rid of a huge part of the tables that can not
be directly allocated using the respective process size.
What are the other pages and allocations that can not rely on the processes size?

One further problem is that deallocation is not working. But in principle, all pages belonging
strictly to one process can be allocated on this sort of stack. The size counter
needs to be adapted to not only account for the processes own memory, but also
for all the memory that is allocated for the process.
\section{Changing physical memory allocation}
\subsection{Special cases}
\paragraph*{Kernel direct mapping}
\paragraph*{MMIO}
\paragraph*{Trampoline}
% \section{Modifying Qemu Source}
% \subsection{TLB Miss Exception}
% \subsection{CSR Modifications}
% \subsubsection{TLB Modification via CSR}
% \subsubsection{Switching between HW and SW TLB miss handling}

% \section{xv6 TLB Miss Handler implementation Idea}
% \subsection{code outline}
% \subsection{Simplest case}
% \subsection{Extrapolation target, most complex case}
% \subsection{features to be implemented}
% \subsection{realisation - how far have I come}

% \section{Implementation description}
% \subsection{Problems}
% \subsection{idiosyncracies}
% \subsection{limitations}

%\section{Veriying the Mapping - Mapping Tester}
\section{Challange: Modifying code in parts}
% in order to test single changes, changes should be isolated to a specific
% kernel module and not change the interface between modules
\subsection*{Changing the physical memory allocation}
% each process gets a fixed set of physical pages assigned to it
% TODO How much processes does xv6 run at a time? -> Least amount of processess to have 
% the operation continue as usual

\section{Simplest Case: Fixed memory area for each process}
\subsection{Address Spaces}
The available physical memory is split into evenly sized portions, from now called address space.
When a new process is started it tries to acquire an address space by checking whether
there is a free one and then claiming that free address space.
The number of address spaces will thus limit the number of concurrently running programs.
Why not just use the Pids and reuse them? The order of termination is not clear, bookkeeping
may be expensive. PID range is typically a lot bigger -> But this can be limited.
\subsection{Discussion: Limitations}
This is not really virtual memory and works only, because xv6 just added more memory at the low
end of the address space. No mappings are made in between or at high addresses.
This is where a hash function could come in handy. However collisions and access rights and
collision detection \ldots
Verification via % \texttt{\verbatim{readelf -l user/_*}}
shows that all vaddr are in the low addresses.
\paragraph*{Fragmentation}
%Processes have fixed size of memory and may not use all of it
\paragraph*{Limited Process count}
%Maybe appropiate for embedded applications?


\section{Debugging}
\subsection{ GDB - custom satps}3